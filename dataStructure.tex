\chapter{除了很強,只有更強}
	\section{讓資料結構如虎添翼!}
		前面學過了許多的資料結構,像是線段樹、Treap、Fenwick Tree(只是想要讓你們複習一下BIT的另外一個名字)等,雖然可以做一些事,但是還是有許多的不足!可以透過新的概念來幫資料結構們加上裝備,讓他們可以做更多事!這裡的東西會比較難理解,需要慢慢讀文字,仔細思考,一小段可以看一個禮拜,慢慢吸收,慢慢理解。
	\section{懶惰標誌(Lazy Propagation)}
		\subsection{就是他!讓我TLE!}
			先來看一個經典題,來展現那些資料結構的不足!
			\problem{經典題(帶修改區間極值)}{
				給定一個長度為$N$的序列,值分別為$a_1, a_2, \dots, a_N$,有$Q$個操作,每一個操作都是以下的其中之一:
				\begin{enumerate}
					\item \inline{1 l r k},代表在$(l, r)$間的所有值都加$k$
					\item \inline{2 l r},請輸出$(l, r)$間所有的值的最大值
				\end{enumerate}
				($N \leq 10^5$,$Q \leq 10^4$)
			}
			不難想到用一個線段樹或維護,這樣子詢問$O(Q_1 \log N\cdot N + Q_2 \log N)$,就爆掉了!所以一定需要更強大,更好的方法來
		\subsection{資料結構的第一個武器——懶標}
			我們做事要懶惰一些,幫電腦省事:如果要加的時候發現這整個區間都需要加,那是不是可以直接記錄下來說這個區間加某個數字$k$,以後如果需要查詢這個區間全部的時候,就直接用數學計算就好了,不需要遞迴下去將每一個點弄好呢?沒錯!就是這樣!
			
			那,具體要怎麼實作呀?對於每一個節點,除了維持值、左右方為何之外、還需要維持一個\inline{lazy}值,代表這個區間每一個值都要加\inline{lazy},如果為$0$則代表目前不需要加。為了實施憲法中的明確性,先來看一下我們所用的\inline{node}長什麼樣吧!
			\begin{C++}
struct Node{
	int l, r, val, lazy; //代表這個節點的左,右界、目前的值、和最新的lazy值!
	Node *left, *right; //指向的子節點
}
			\end{C++}
			\subsection{兩把小刀:\inline{push} 和 \inline{pull}}
				對於每一個\inline{node},都應該要有兩個函數,\inline{pull}應該已經看過了,很簡單:
				\begin{C++}
void pull(Node* n){
	n->val = n->left->val + n->right->val;
}
				\end{C++}
				但是最新進來的捧油\inline{push}就比較奇特一點了,因應打懶標而出現,也就是將懶標往下打一層,話不多說,直接看程式!
				\begin{C++}
void push(Node *n){
	if(!n->lazy) return; //不需要做事
	n->val += n->lazy; //若整個區間都要加lazy,則最大值也會加lazy
	if(n->l + 1 < n->r){ //如果還有在下面的區間,則打下去,注意沒有遞迴!
		n->left->lazy += n->lazy;
		n->right->lazy += n->lazy;
	}
	n->lazy = 0;
}
				\end{C++}
	\subsection{原本的函數要進化了!}
		現在來示範怎麼修改原本有的\inline{query}函數和\inline{modify}函數:
		\subsubsection{新的\inline{query}}
			簡單來說,就是在詢問之前,都要確定沒有懶標需要\inline{push}了,再查詢!
			\begin{C++}
int query(Node *n, int ql, int qr ){
	push(n); //重要重要重要!
	if(ql >= n->r || qr <= n->l) return -INF; //出界
	if(ql <= n->l && n->r <= qr) return n->val; //完全包含
	return max(query(n->l, ql, qr), query(n->r, ql, qr));
}
			\end{C++}
		\subsection{新的\inline{modify}}
			其實與\inline{query}差不多,先記得\inline{push}之後,修改懶標即可。
			\begin{C++}
void modify(Node *n, int ql, int qr, int val){
	push(n); //重要重要重要!
	if(ql >= n->r || qr <= n->l) return; //出界
	if(ql <= n->l && n->r <= qr){ //完全包含
		n->lazy += val; 
		return;
	}
	modify(n->l, ql, qr, val); 
	modify(n->r, ql, qr, val);
	pull(n); //別忘了
	
}
			\end{C++}
		這就是你的第一個(或是第$k$個,$k \in \mathbb{N} \cap \{0\}$)懶標線段樹了!重點就是,如果遇到一個節點,可以盡量不要修改就不要修改,只是記錄下來,到了真的需要修改再修改之。
	\section{另外一種寫法}
		會有人覺得,修改的時候沒有修改到東西不夠爽,所以就出現了另外一種寫的方法,可以供參考:想法就是,\inline{lazy}代表\textbf{子樹}所需要修改的東西,自己則是修改完了!這樣,\inline{push}會變成
		\begin{C++}
void update(Node *n, int val) { //helper function
	n->v += val, n->lazy += val;
}
void push(Node *n){
	if(!n->lazy || n->l+1 >= r) return;
	update(n->left,n->lazy);
	update(n->right,n->lazy);
	n->lazy = 0;
}
		\end{C++}
	至於\inline{modify},也會變乾淨:
		\begin{C++}
void modify(Node *n, int ql, int qr, int val) {
	push(n); //重要重要重要!
	if(ql >= n->r || qr <= n->l) return; //出界
	if(ql <= n->l && n->r <= qr){ //完全包含
		update(n, val); 
		return;
	}
	modify(n->l, ql, qr, val); 
	modify(n->r, ql, qr, val);
	pull(n); //別忘了
}
		\end{C++}
	\subsection{習題}
		來練習一下新學到的技巧吧!
		\problem{Ahoy Pirates! (UVa 11402)}{
			給定一個長度為$N$,且由$0$和$1$組成的序列,請支援$Q$個操作,每一個都是以下操作之一:
			\begin{enumerate}
				\item \inline{F a b}:請把第$a$個到第$b$個位置的數字都變成$1$
				\item \inline{E a b}:請把第$a$個到第$b$個位置的數字都變成$0$
				\item \inline{I a b}:請把第$a$個到第$b$個位置的數字,$0$變成$1$,$1$變成$0$
				\item \inline{S a b}:請輸出第$a$個到第$b$個位置的數字總共有幾個$1$
			\end{enumerate}
			($N \leq 10^6$,$Q \leq 10^5$)
		}
		\problem{Circular RMQ(CF 52C)}{
			給定一個長度為$N$的環形(也就是最後一項和第一項相鄰)序列$a_0, a_1, \cdots, a_{N-1}$,和有$Q$筆操作,都是以下的兩個其中之一:
			\begin{enumerate}
				\item $\text{inc}(lf, rg, v)$:將$[lf, rg]$內的數字全部加$v$
				\item $\text{rmq}(lf, rg)$:請輸出$[lf, rg]$中最小的數字
			\end{enumerate}
			($1 \leq N \leq 200000$,$0 \leq Q \leq 200000$,$|v|, |a_i| \leq 10^6$)
		}
		\problem{矩形覆蓋面積計算(TIOJ 1224)}{
			很經典的一題:給你平面上$N$個矩形,請求那些矩形所覆蓋出來的面積為何?(如果多個矩形蓋到同一個地方,只能算一次)
			$N \leq 10^5$,且矩形的$x, y$座標皆為在$0$和$10^6$之間的整數。
		}
		\problem{《Φ》序章·IV ~ 生活作息(ZJ c251)}{
			給定一個長度為$N$的序列$S$,有$Q$次如下的操作:
			\begin{enumerate}
				\item \inline{0 L R}:輸出$[L,R]$中有幾種不同的數字
				\item \inline{1 L R P}:把$[L,R]$中所有數字修改為$P$
			\end{enumerate}
		($Q, N \leq 2^{15}$,$ S_i \leq \min(N,2^5)$,
		輸入有至多$2^5$筆測資)
		}
	\section{持久化資料結構(Persistent Data Structure)}
		\subsection{會不會Cmd + Z啊?}
			有時候,在修改資料結構的時候,會想要同時保存舊的版本和新的版本,但是如果整個再複製一次,這樣子記憶體會用的很兇(又不是Google Code Jam給10GB!),所以必須用到持久化的概念來將記憶體壓回正常的範圍。
		\subsection{什麼是持久化?}
			持久化的精神就是:\textbf{每次修改之後,只新複製出被動到的地方,其餘都和原本的共用},如果不清楚,就看圖吧!如果太多搞不清楚了,就看天吧!
			\begin{center}
				\includegraphics*[width = 0.7\textwidth]{pictures/dataStructure/persistent}
			\end{center}
			這裡,我們要對原本的線段樹(Version 0)做修改,而我們要修改編號為$13$的節點。所以呢,沿路會動到$1, 3, 6, 13$這四個節點,只需要新建立出$1', 3', 6', 13'$這四個節點成為Version 1,其他的就可以與舊的Version 0共用了!這樣,新增的空間複雜度就從原本的$O(n)$變成$O(\log n)$了!
		\subsection{實作方法 —— 來種一棵持久化線段樹吧!}
			其實重點就是要存那些新的根節點,用一個陣列存就可以了!先來看\inline{struct Node}長什麼樣子吧!
			\begin{C++}
struct Node{
	static int size; //整個線段樹所代表的範圍有size這麼大
	int val;
	Node *l, *r;
	Node(int val): val(val), l(nullptr), r(nullptr){}
	Node(Node *l,Node *r): l(l), r(r){ pull(); }
	void pull() { if(l&&r) val = l->val + r->val; }
};
int Node::size; //static member 必須這樣宣告!否則吃CE
			\end{C++}
			 除了\inline{modify}以外的函數都和原本的線段樹差不多,可以明顯的看到,我們在做修改的時候都是新建一個\inline{Node},而不會動到原本的!這樣就可以完整的保存修改的歷史(悠久的文化呀)版本了!
			 \begin{C++}
Node *modify(Node *o, int pos, int dif, 
					   int l=0, int r = Node::size){ 
	if(l+1 >= r) return new Node(o->val + dif);
	int mid = l + (r-l>>1); //好習慣,防止(l + r)溢位
	if(pos < mid)
		return new Node(modify(o->l,pos,dif,l,mid), o->r); 
	else 
		return new Node(o->l, modify(o->r,pos,dif,mid,r));
	//其中一個變成新的,另外一個不用修改,指回原本的
}
			 \end{C++}
			剩下的就是一般的線段樹就好了(至少會和講義中的函數差不多),不需要變!每一次\inline{modify}的時候,就將回傳的新\inline{Node}小心翼翼地將它放進去存的地方,如陣列或BIT(真的)等。
		\subsection{其他資料結構的持久化}
			不只線段樹可以持久化,只要你想要,都可以持久化,只是比較難:幾乎所有的樹狀結構都可以持久化,像是Treap、Trie、左偏樹(可合併的heap)、Linked List、並查集(當然,就不能路徑壓縮了,複雜度退化成$O(\log n)$)等都可以寫成持久化,甚至序列看成Treap也可以持久化!可以去看CodeChef上有一篇文章叫做「\href{https://discuss.codechef.com/t/persistence-made-simple-tutorial/14915}{\underline{Persistence Made Simple}}」(URL),寫得很好,將持久化的精神解釋的很清楚。