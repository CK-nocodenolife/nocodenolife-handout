\chapter{字串\uppercase\expandafter{\romannumeral 2}}
\section{前言}
		前面已經學過了關於字串的一些知識,譬如KMP演算法,字典樹(Trie)等,這裡會介紹一些更進階的資料結構與演算法來更有效率的處理字串的各種問題,其中非常精妙,值得細嚼慢嚥的讀。
	\section{後綴序列Suffix Array}
		\subsection{什麼是後綴序列?}
			要看後綴序列之前,以防混淆,先定義一下字串的後綴是什麼:
			\definition{後綴}{
				首先,來定義後綴:對於一個字串$S[0\dots n - 1]$,定義其第$i$個後綴為$S[i \dots n - 1]$。可以知道,第$0$個後綴為字串本身,而第$n - 1$個字串就是$S$的最後一個字元。	
			}
			知道後綴是什麼了,就可以定義後綴序列了:
			\definition{後綴序列}{
				對於一個字串$S$,我們將其後綴照字典序排序之後,將後綴序列$SA[i]$定義為「$S$的字典序第$i$小的後綴是第幾個後綴」
			}
			或許有一點抽象,所以我們就來跑一次例子試試看吧!令$S = abaab$,我們可以得到其後綴們:
			\begin{center}
				\begin{tabular}{c  c}
					0&$abaab$\\
					1&$baab$\\
					2&$aab$\\
					3&$ab$\\
					4&$b$\\
				\end{tabular}
			\end{center}
			大家一定都想得到一個很簡單的方法來建這個序列:把東西全部都放進去一個陣列之後\inline{sort},但是因為兩個字串的比較是$O(N)$($N$為字串長度),所以這樣複雜度是$O(N^2 \log N)$,如果稍微大一點的字串就會超時。在這裡,我們介紹一個$O(N \log^2 N)$的演算法,雖然有$O(N \log N)$甚至$O(N)$的演算法,但是都沒那麼好寫,比賽通常也不會壓這個常數,所以先介紹這個簡單一點的版本。
		\subsection{建立後綴序列}
			注意:這裡要排序的一堆字串,不是隨便抓來的隨機字串,而是同一個字串的後綴!我們可以利用這個性質,來對演算法優化。我們的中心概念是「倍增法」:若可以依照前$2^k$個字元排序,能否用這個資訊得到前$2^{k + 1}$個字元的排序?答案是:可以的。假設對於每一個位置$i$,都已經知道對於目前前$2^k$個字元來說,$i$是第幾位(當然如果相同的話就並列),那要怎麼得到下一個是第幾位?
			
			其實很簡單:注意到第$i$個前綴的前$2^{k}$個字元其實就是第$i$個前綴的前$2^{k - 1}$個字元,後面加上第$i + 2^{k - 1}$個前綴的前$2^{k - 1}$個字元。所以,要比較第$i$個前綴的前$2^{k}$個字元和第$j$個前綴的前$2^{k}$個字元那一個比較大就可以用前面所計算的來比較了。
			
			\definition{Rank Array}{
				有些人會叫做$SA^{-1}$,在此稱作$rank$,也就是$SA$的「反函數」,$SA$是給定一個「第幾名」,它跟你說第幾名是誰;而我們這個$rank$陣列則是:給定第$i$個後綴,它會告訴我們它是第幾名。以數學方式來說,對於$0 \leq i < N$,$rank$會滿足:
				$$rank[SA[i]] = i$$  
			}
			
			具體來說,若$L[i][k]$代表第$i$個前綴的前$2^{k}
			$個字元(若沒有則取代為最小字元)是第幾小的。我們想要由$L[i][k]$獲得$L[i][k + 1]$。則因為第$i$個前綴的前$2^{k}$個字元其實就是第$i$個前綴的前$2^{k - 1}$個字元,我們如果要看第$i$個前綴和第$j$個前綴的前$2^{k + 1}$個字元哪一個比較小,先比較前面的:$L[i][k]$和$L[j][k]$。若分不出高下,再比較$L[i + 2^{k}][k]$和$L[j + 2^{k}][k]$,若超出範圍則看$i$和$j$哪一個比較大,大的代表字串長度比較小,放前面。以上會用到$O(N \log N)$的空間,可以滾動掉。時間的部分,可以知道有$O(\log N)$次的計算,每次都需要$O(N \log N)$來排序,所以總時間複雜度$O(N \log^2 N)$。程式如下:
			\begin{C++}
vector<int> SA, nextrk, rk; //目前的後綴序列,下一個Rank,目前的Rank(第幾個)
int gap, N;

bool cmp(int i, int j){
	if(rk[i] != rk[j]) 
		return rk[i] < rk[j]; //若前面就能比較
	
	i += gap;
	j += gap;
	
	if(i < N && j < N) 
		return rk[i] < rk[j]; //比較後面
	else 
		return i > j; //字串小的放前面
}

void getSA(string s){
	N = s.length();
	SA.resize(N);
	nextrk.resize(N);
	rk.resize(N);
	
	nextrk[0] = 0;
	
	for(int i = 0; i < N; i++){ //初始化,待會就會sort掉了
		SA[i] = i;
		rk[i] = s[i] - 'a';
	}
	
	for(gap = 1; ; gap *= 2){ //gap就是一次看幾個字串
		sort(SA.begin(), SA.end(), cmp);
		for(int i = 1; i < N; i++) 
			nextrk[i] = nextrk[i-1] + cmp(SA[i - 1], SA[i]); //若不同了,就要加一
		for(int i = 0; i < N; i++) 
			rk[SA[i]] = nextrk[i]; //更新Rank
		if(nextrk[N - 1] == N - 1) break; //結束條件
	}
}
			\end{C++}		
		\subsection{$O(N \log N)$建立後綴序列}
		嗯,剛剛的$quick sort$還真的不錯,但畢竟時間複雜度很難唸,而且複雜度為$O(N \log N)$的解也沒有難(長)太多,所以就不學白不學囉!其實做法跟剛剛所敘述的基本上一樣,也是使用倍增法,每次排序一次,只是真的有必要每次都做一次快速排序嗎?
		
		等等,快速排序應該是比較行排序中數一數二快的了吧,還有甚麼排序法比快速排序法更快呢?
		
		大家可能會想到所謂的計數排序(counting sort)可以$O(N+C$(值域))做完排序。那大家不妨想想,這裡每次要排序時的值域多大呢?因為總共要比較兩次$rk$,而所有後綴的$rk$最多有$N$種可能,所以這裡的值域是$O(N^2)$,空間和時間都不允許。
		
		那還有甚麼排序方式呢?如果要排序值域界在$1$到$100$之間的整數,真的只能開$100$格來做counting sort嗎?有沒有可能只開十格或二十格?其實有一種作法叫做radix sort,不知道大家知不知道?假設要排序的數值都是以十進位制表示的二位整數,我們就開兩個十個格子的陣列$A$、$B$,先做個位數再十位數。第一次把每個整數$a_i$丟到$A[a_i\%10]$。第二次的時候從$A[0]$的最底部(最早被丟進去的)開始往上看,再依序看$A[1]$到$A[9]$,也都是由底部往上看。這次,依照被看到的順序將每個整數$a_i$丟到$B[a_i/10]$。做完之後,從$B[0]$的最底部開始往上、往右看,將看到的數字依序記錄下來,就是排序好的序列了。大家可能不太懂,可以參考這個動畫:\url{https://www.youtube.com/watch?v=xuU-DS_5Z4g}。這真的是一個很神奇的方法,我一開始聽到的時候也沒有完全理解他為甚麼會對,但是自己動手畫畫看,就應該可以理解其中的奧妙了。它的複雜度應該是$O(($開的格字數$+$序列長度$) \times$進行輪數$)$。
		
		好,說了這麼多,不知道大家有沒有了解為什麼在這裡上radix sort會使複雜度變好。那我們就將目前的狀況代入它的複雜度中吧!好,首先序列長度不用說,就是字串長度$N$嘛,而進行輪數就是$2$(比較前半段和後半段),那我們要開幾格呢?因為每一輪都只要依照前半段或後半段的$rk$將其塞到對應的格子中,$rk$右最多只有$N$種,所以只要開$N$個格子就好了!因此複雜度是$O((N+N) \times 2) = O(N)$,就這樣,一個$\log$就被壓掉了!
		
		但實做上還是有一些技巧可以使用,就先煩請大家先稍微看一下code有那裡不懂,在code後我會再詳細解釋。
		\begin{C++}
#define dictsz 200000
//這裡假設有200000種字元
int gap, N;
vector<int> SA, nextrk, rk, temp, cnt;
bool cmp(int i, int j){
	if(rk[i] != rk[j])
		return rk[i] < rk[j]; //若前面就能比較

	i += gap;
	j += gap;
	
	if(i < N && j < N)
		return rk[i] < rk[j]; //比較後面
	else
		return i > j; //字串小的放前面
}

void getSA(string s){
	N = s.length();
	SA.resize(N);
	nextrk.resize(N);
	rk.resize(N);
	cnt.resize(dictsz);//這就是桶子啦
	nextrk[0] = 0;
	int i;
	for(i = 0; i < N; i++) //初始化,待會就會sort掉了
		rk[i] = s[i];
	for(i = 0;i < dictsz;i++)
		cnt[i] = 0;//初始化
	for(i = 0;i < N;i++)
		cnt[rk[i]]++;//丟到桶子
	for(i = 1;i < dictsz;i++)
		cnt[i] += cnt[i-1];//待會解釋
	for(i = N-1;i >= 0;i--)
		SA[--cnt[rk[i]]] = i;//待會解釋
	for(gap = 1;; gap *= 2){ //gap就是一次看幾個字元
		temp.clear();
		for(i = N-gap;i < N;i++)
		temp.push_back(i);//後半段會超出的後綴後半段一定最小字典序
		for(i = 0;i < N;i++){
			if(SA[i] >= gap)
			temp.push_back(SA[i]-gap);//將後綴沒有超出的後綴按照字典序丟進去
		}
		//注意,這裡的temp就會依照所有後綴後半段的字典序由小到大排列了
		for(i = 0;i < dictsz;i++)
			cnt[i] = 0;//把桶子清空
		for(i = 0;i < N;i++)
			cnt[rk[temp[i]]]++;//照順序將每個後綴丟到代表其前半段字典序大小的桶子中
		for(i = 1;i < dictsz;i++)
			cnt[i] += cnt[i-1];//做前綴,為甚麼?待會解釋
		for(i = N-1;i >= 0;i--)
			SA[--cnt[rk[temp[i]]]] = temp[i];//這啥?待會解釋
		nextrk[SA[0]] = 0;
		for(i = 1; i < N; i++)
			nextrk[SA[i]] = nextrk[SA[i-1]] + cmp(SA[i - 1], SA[i]); //若不同了,就要加一
		for(i = 0; i < N; i++)
			rk[i] = nextrk[i]; //更新Rank
		if(rk[SA[N - 1]] == N - 1) break; //結束條件
	}
}
		\end{C++}
		
		大家應該可以發現,code跟之前長得差不多。我覺得大家最不能理解的應該就是為甚麼要做前綴,還有這句\inline{SA[--cnt[rk[temp[i]]]] = temp[i]}到底是什麼意思吧?其實這兩件事是一體的。簡單來說,做前綴的目的就是讓要放進去的時候知道要放到$SA$中的哪個位子。再仔細看看上面那行,它的\inline{for}迴圈其實是倒過來寫的,它就是由後往前掃過一遍已經按照後半段排好的後綴,然後$cnt$的前綴值減一不就是它應該放在$SA$中的位置嗎?
	\section{最長共同前綴序列LCP Array}
		\subsection{LCP Array又是什麼?}
			LCP Array是有了後綴序列之後的一個好幫手,可以再次加速演算法!這一次也要先定義所需要的函數:對於兩個字串$A$和$B$,我們定義$lcp(A, B)$為他們兩個的最長共同前綴的長度。
			\definition{LCP Array}{
				對於一個字串$S$和其前綴序列$SA$,我們定義其LCP陣列$LCP[i] = lcp(T_{SA[i]}, T_{SA[i - 1]})$,且$LCP[0] = 0$。此處$T_i$代表$S$的第$i$個前綴。
			}
			當然,不難想到一個$O(N^2)$的計算方法,但是這樣一定太慢!有一個$O(N)$的計算方法,讓我們繼續看下去!
		\subsection{Kasai's Algorithm}
			這個神奇的方法稱為Kasai's Algorithm,能夠在給定字串$S$和其後綴序列$SA$的情況下在$O(N)$算出LCP陣列。首先,我們先定義一個東東:
			
			接下來,這個部分可能你會需要重讀很多遍,因為真的蠻細的;不過搞懂之後,實作不複雜。
			
			Kasai's Algorithm的中心思想在於思考第$i$個後綴與第$i + 1$個後綴的關係:因為只是拿掉前面一個字元,所以會有好性質可以用!
			
			首先,考慮後綴序列裡面的連續兩個位置,稱為$i_1$與$i_1'$。若$lcp(i_1, i_1') = 0$就算了;否則我們可以知道,$T_{i_1}$和$T_{i_1'}$的第一個字元一定相同。所以,如果拔掉第一個字元會怎麼樣?你會發現,$T_{i_1}$拔掉第一個字元不就是$T_{i_1 + 1}$嗎?首先,我們可以知道$lcp(i_1 + 1, i_1' + 1) = lcp(i_1, i_1') - 1$,因為前者只是後者拿掉第一個字元而已,所以顯然是對的。
			
			因為$lcp(i, j) = \min(lcp_i, lcp_{i + 1}, \dots, lcp_{j})$(左邊的$lcp(i, j)$代表$T_i$和$T_j$的最長共同前綴的長度),我們可以知道$lcp(i_1 + 1) \geq lcp(i_1 + 1, i_1' + 1) =lcp(i_1, i_1') - 1$,所以只要從上一個值減一(若不是零的話)計算就好了。注意:這裡有兩個序列:第一個是後綴序列的順序,第二個是原本後綴的順序,此處$T_k$都是代表第$k$個後綴。而因為計算第$T_i$的時候會用到$T_{i -1 }$,所以跑的順序是依照後綴的順序跑的,而不是依照後綴序列的順序跑。程式:
			\begin{C++}
vector<int> SA, lcp;
void getlcp(string s){
	lcp.resize(N);
	vector<int> SAinv;
	SAinv.resize(N);
	for(int i = 0; i < N; i++){
		SAinv[SA[i]] = i;
	}

	int k = 0;
	for(int i = 0; i < N; i++){
		if(!rk[i]){ //若其為第一個,則沒有lcp
			k = 0;
			continue;
		}
		int j = SA[rk[i] - 1]; //拿到後綴序列的上一個
		while(i + k < N && j + k < N && s[i + k] == s[j +k])  		
			k++;  //利用上面的性質硬跑
		lcp[SAinv[i]] = k; //設定,不是lcp[i]!
		if(k) k--; //界是 -1
	}
}
			\end{C++}
		而這個為什麼是$O(N)$呢?因為\inline{k} 除了一次之外,每次都最多減少$1$,所以\inline{while} 迴圈不可能跑太多次,故時間複雜度是$O(N)$。

\section{AC自動機(AC Automaton)}
AC自動機顧名思義就是一種可以讓你自動AC的程式啊(X\\

好,絕對不要聽別人這樣唬爛,事實上AC自動機是KMP字串匹配演算法的進階版,可以一次尋找很多目標字串各個在主字串中出現的地方。不知道大家還記不記得,KMP演算法是在進行字串匹配前,計算當匹配失敗時,可以直接從哪個位置開始重新匹配,也就是failure function。而在AC自動機中也是利用相同的手法,在目標字串們建構出的trie上計算failure pointer,同樣代表匹配失敗時應從哪裡重新開始。
\subsection{節點}
AC自動機的節點大致上與trie都一樣,只是會多新增幾個資料,詳細的\inline{struct}如下:
\begin{C++}
struct node{ //K是字元集的大小
	node *ch[K],*fail; //記錄子節點們和failure pointer
	vector<int> endof; //紀錄這個節點表示了那些字典集中的字串
	node(): endof(){
		for(int i=0; i<K; i++) ch[i]=nullptr;
		fail=nullptr;
	}
};
\end{C++}
\subsection{build AC automaton}
最重要的當然是先建構出trie才能做其他事吧!當然,建構方式和正常的trie也差不多,因此就直接看code吧:
\begin{C++}
node *build(const vector<string> &dict) {
    node *root=new node();
    for(int i=0; i<dict.size(); i++) {
		node *now = root;
		for(char c: dict[i]) {
			if(!now->ch[c-'a'])
				now->ch[c-'a'] = new node();
			now = now->ch[c-'a'];
		}
		now->endof.push_back(i);
    }
    return root;
}
\end{C++}
\subsection{failure pointer}
原本failure function代表的是某個前綴的「次長共同前後綴」,在trie上沿用這個概念的failure pointer指向的是「次長的存在於這個trie中的後綴」,我們同樣能知道failure pointer肯定離根比較近,因此移到trie時我們選用BFS的遍歷順序依序計算,如此便可保證計算某一節點時,長度更短的都已經算完了。說了這麼多,不如直接看程式碼吧:
\begin{C++}
vector<node *> buildFail(node *root) {
	vector<node *> BFS; // 儲存BFS的遍歷順序,待會用到
	queue<node *> Q;
	for(int c=0; c<K; c++) {
		if(root->ch[c]) {
			root->ch[c]->fail = root;
			Q.push(root->ch[c]);
		}
	}
	while(!Q.empty()) {
		node *p = Q.front(); Q.pop();
		BFS.push_back(p);
		for(int c=0; c<K; c++) if(p->ch[c]) {
			node *f = p->fail;
			while(f != root && !f->ch[c]) f = f->fail;
			if(f->ch[c]) f = f->ch[c];
			p->ch[c]->fail = f;
			Q.push(p->ch[c]);
		}
	}
	return BFS;
}
\end{C++}
\subsection{開始匹配}
不外乎就是跟KMP的方法一樣,從第一個字開始看,若遇到失配的狀況就走failure pointer。程式碼應該也相當易懂:
\begin{C++}
void match(node *root, const string &S, const vector<string> &dict){
    node *p = root;
    for(int i=0; i<S.size(); i++) {
    	while(p != root && !p->ch[S[i]-'a']) p = p->fail;
    	if(p->ch[S[i]-'a']) p = p->ch[S[i]-'a'];
    	for(int id: p->endof) {
    		cout << "Match pattern #" << id << " at position"
    				<< i - dict[id].size() << '\n';
    		}
    }
}
\end{C++}
\subsection{你以為結束了?}
多玩幾次之後就會發現上面AC自動機在目標字串互相包含時會出問題!看下面範例:\\

\begin{center}
dict=\{"ab","cababc"\} \\
S="cababc"
\end{center}

上面程式的輸出結果"ab"消失了!為什麼會這樣?你畫出AC自動機後自己走一次就會發現,整個匹配過程都是走"cababc"的那條路,然而經過"ab"時,完全沒有標記讓AC自動機知道他也是一個單字。也就是說,只要某個字被包含在另一個字的中間時,就會被忽略。怎麼辦呢?\\

假設從節點$S_{0,1\dots k-1}$走到$S_{0,1\dots k}$時,會使$S_{1,2\dots k-2}$的所有後綴字串沒有被考慮到。因此我們應該要窮舉這些字串,檢查他們是不是也在字典中。然而,我們可以發現$S_{1,2\dots k-2}$後綴字串中所有在字典裡的字串,事實上就是不斷走failure pointer路上經過的那些字串!想一想就知道了!\\

\begin{C++}
void match(node *root, const string &S, const vector<string> &dict){
    node *p = root;
    for(int i=0; i<S.size(); i++) {
		while(p != root && !p->ch[S[i]-'a']) p = p->fail;
		if(p->ch[S[i]-'a']) p = p->ch[S[i]-'a'];
		for(node *now=p; now!=root; now=now->fail) { //跳fail
			for(int id: now->endof) {
				cout << "Match pattern #" << id << " at position" << i - dict[id].size() << '\n';
			}
		}
	}
}
\end{C++}

如果只想知道各個字串被匹配到的次數,我們可以記錄所有點被走過幾次,然後在Match跑完之後將每個節點被經過的次數不斷向他的failure pointer加,類似在樹或DAG上DP的感覺。那應該用什麼順序處理這件事呢?因為要不斷向上推,不如就從最深的開始處理,由於failure pointer的深度一定比該節點的深度小,直接用BFS遍歷順序的逆序正好可以保證正確性!直接來看一下code吧!假設node裡面已經有宣告了\inline{cnt},並且假如大字串在match中每一步走到的每個節點\inline{cnt}都增加1,經過DP之後最後的值將會表示大字串和這個節點所代表的字串的匹配數量。
\begin{C++}
void match(node *root,const string &S){
    node *p = root;
    for(char c: S) {
    	while(p != root && !p->ch[c-'a']) p = p->fail;
    	if(p->ch[c-'a']) p = p->ch[c-'a'];
    	p->cnt += 1;
    }
    reverse(BFS.begin(), BFS.end());
    for(node *p: BFS) {
		for(int id: p->endof) {
    			cout << "Match pattern #" << id << " " << p->cnt << " times\n";
		}
		p->fail->cnt += p->cnt;
    }
}
\end{C++}

\subsection{一個小優化}
看看我們的\inline{match}函式可以發現,如果我們要match很多個(或很多次)大字串,每次都得跑failure,感覺很浪費時間呢!是不是可以預先處理好如果以某個字元失配會跳到哪個節點呢?答案是肯定的!這樣做的優點還有在AC自動機上來回走的時候保證是$O(1)$,而跳fail的方法僅保證一次匹配的均攤是好的,不能保證來回走的時候不會跳太多次(後面有習題似乎需要這個優化,嘻嘻)。下面的code直接用\inline{ch}來儲存一個節點遇到一個字元應該跳到哪個節點,若會需要用到這個trie原來的結構可以另外開一個空間儲存。

\begin{C++}
void buildFail(node *root) {
	queue<node *> Q;
	for(int c=0; c<K; c++) {
		if(root->ch[c]) {
			root->ch[c]->fail = root;
			Q.push(root->ch[c]);
		} else root->ch[c] = root; // base case很重要
	}
	while(!Q.empty()) {
		node *p = Q.front(); Q.pop();
		for(int c=0; c<K; c++) {
			node *f = p->fail;
			if(p->ch[c]) { // 沒失配繼續往下走
				p->ch[c]->fail = f->ch[c];
				Q.push(p->ch[c]);
			} else {
				p->ch[c] = f->ch[c]; // 失配跳fail
			}
		}
	}
}
void match(node *root, const string &S) {
	node *p = root;
	for(int i = 0; i < S.size(); i++) {
		p = p->ch[s[i]-'a'];
		// do something on node p
	}
}
\end{C++}
\subsection{更多活用}
這邊有一個引理
\lemma{樹狀結構}{
所有failure pointer會形成一棵樹,因為恰好有(節點數-1)條指向別人的failure pointer,並且每個節點的failure pointer唯一且比自己淺,所以不會有環。
}
有些題目會要求在AC自動機上動態一些的做事,因為一個節點可能對他的祖先或是整個子樹產生影響,我們可以用樹壓平解決。

AC自動機另一個優點是,他把所有可能跟小字串匹配的狀態從$\prod |P|$壓到$\sum |P|$(原本紀錄每個字串match到第幾個,現在只要知道match到哪個節點),因此也常常出現和DP配合的題目,例如筆者寫這份講義左右的入營考最後一題。

%\section{Burrow-Wheeler Transform}
%	\subsection{這是啥?有什麼用?}
%		因為在2019的YTP複賽出現過,所以就來介紹一下這個是做什麼的;//TODO
%		\theorem{abc}{def}


\section{習題}
以下題目有可能不一定要用到SA或AC自動機,請自行發揮創意找出更多不同解法!
\problem{似曾相識 (TIOJ 1515)}{給定一字串$s$,請問在此字串中重複出現兩次以上的最長字串長度為何(若無則輸出0)?}
\problem{字串中的字串 (TIOJ 1306)}{給你一個字串$T$,以及很多字串$P$。對於每個$P$請輸出$P$在$T$中出現過幾次。($T$、$P$都是由小寫字母所組成,長度$\leq 10^4$)}
\problem{猜謎遊戲 (Guess)(TIOJ 1091)}{
給定$S,P(|S| > |P|)$,請問至少要在$S$中刪除幾個字元才能使得其不包含$P$作為子字串? \\
註:字元僅有AB兩種。$|P| \leq 100, |S| \leq 1000$
}
\problem{Letters and Question Marks(CF 1327G)}{
給一個字串$S$,還有一些小字串$[t_1, t_2, \dots, t_k]$,每個小字串$t_i$有價值$c_i$,同時$S$和$t$除了問號以外都只包含前$14$個英文字母(a到n)。 \\
$S$包含了不超過$14$個問號,這些問號必須被填入兩兩相異的字母,請找出所有可能的填入方法中,總價值最大可以是多少。總價值的計算方法,是對每個小字串計算在該字串中的出現次數乘上該小字串的價值,最後直接相加。 \\
$k \leq 1000, \sum |t_i| \leq 1000, |c_i| \leq 10^6, |S| \leq 4 \cdot 10^5$
}
\problem{Indie Album(CF 1207G)}{
Mishka最愛的樂團發行了新專輯!專輯中包含了$N$首歌,每首歌的歌名$s_i$可能是以下兩種:
\begin{enumerate}
\item $1 \ c$表示$s_i$是一個單獨的字母$c$
\item $2 \ j \ c$表示$s_i$是$s_j$再加上一個字母$c$($j < i$)
\end{enumerate}
Vova對這張專輯很感興趣,但他沒有足夠的時間去聆聽所有歌曲,因此他詢問了Mishka一共$M$個問題,每個問題包含了$i$和一個字串$t$,表示他想知道$s_i$裡面出現了幾次$t$。
Mishka不知道他為何要問這些問題,但他盡力提供訊息。你可以幫助Mishka回答Vova所有的問題嗎? \\
所有字串僅包含小寫拉丁字母,$N, M \leq 4 \cdot 10^5, \sum |t| \leq 4 \cdot 10^5$
}
\problem{同步(Sync)(TIOJ 1927)	}{
在一個多人單向卷軸遊戲中,有$N \leq 10^5$個格子,每個格子都有一個不超過$10^9+6$的正整數,代表該格的狀況。
有時遊戲中的兩人會產生「同步」的現象。產生同步的條件是兩人所在的格子的數字$a,b$分別滿足$(ab)^\frac{p-1}{2} \equiv 1 \pmod p$,其中$p = 10^9+7$。產生同步後,兩人會瞬移至下一格。如果在下一格又產生「同步」,則會繼續往下走,直到其中一人超出格子範圍(到了終點了)或者兩人不再同步。

現在有$Q \leq 10^6$個詢問,每次給定$N$個格子中的正整數以及兩人的起始位子,請計算他們兩個將會「同步」幾次。
}