\chapter{根號算法}
\section{前言}
	「十則圍之、五則攻之、倍則分之」,這是孫子兵法中提到的,說要視面對敵人的多寡,要採用不同的策略。同樣地,如果題目給的範圍有不同的限制,可能需要不同的作法。有時,這個限制可可能會在題目中就給了;不過,有時候題目連這個都不會給!所以,就必須要自己來界定分離的標準了。這個方法有時候看起來很通靈,甚至很唬爛,但是經過分析之後是對的,時常能將複雜度的一個變數降為其根號,譬如一個$O(N^2)$的演算法其實經過一點巧思是$O(N\sqrt{N})$,雖然會比$\log$等級的演算法慢一些,但是通常實作複雜度低許多,是一個很實用的工具!