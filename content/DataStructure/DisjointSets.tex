\documentclass[main.tex]{subfiles}
\begin{document}

\chapter{並查集 Disjoint Sets}
	並查集 (Disjoint Sets)是一個用來處理集合的資料結構,他只支援兩種操作:
	\begin{enumerate}
		\item 將兩個集合合併(Union)
		\item 查詢一個元素所在的集合(Find)
	\end{enumerate}
	這是一個基礎的資料結構,一定要學會,後面的演算法常常會用到這個概念!
		
	\subsection{實作}
	一般我們維護一個DSU的方式都是以一棵樹來記錄一個集合,而利用root來代表該集合,事實上,要維護這些樹只要用一個array就行了,array第$i$項的值就是$i$的parent,而root的parent則是自己。前面說到一個正常的DSU應該要支援兩種功能,現在就來講講這兩種功能的實作概念吧~
	\subsection{合併(Union)}
	合併兩個集合的方式其實異常簡單,就是分別找到那兩個集合的root,將其中一個root指向另外一個,便大功告成了。
	\subsection{查詢(Find)}
	要查詢一個元素所在集合的方式也十分直觀:不斷往自己的parent找,直到找到一個元素的parent指向自己,那麼他就是root了。而在往上尋找時還有一個小技巧,就是在過程中記錄經過的節點,之後一併將這些點的parent改為root,便可以節省下次查詢這些點的時間。
	\problem{DSU 練習}{
		請實作一個\inline{DSU},需要支援兩種操作:
		\begin{enumerate}
			\item \inline{Union}$(a, b)$,將 $a$ 所在的集合與 $b$ 所在的集合合併
			\item \inline{Find}$(a)$,回傳 $a$ 所在的集合
		\end{enumerate}
		}
	\begin{C++}
int v[MAXN];

void Find(int a) {
	if (v[a] == a) return a;
	v[a] = Find(v[a]);
	return v[a];
}

void Union(int a, int b) {
	v[Find(a)] = Find(b);
}
	\end{C++}
	\subsection{維護 size}
	倘若想同時記錄每個集合的大小,其實也沒那麼困難,只要多用一個size變數紀錄每個集合的大小,合併時將兩個集合的大小相加,就能成功維護集合大小了。
	\subsection{一點小優化}
	對於每個集合多給一個變數rank,記錄每個集合中tree的最大深度,合併時選擇rank較小的合併到rank較大的集合中,而倘若兩個集合的rank相同,則要在兩個集合合併後把rank值$+1$,這樣可以再降低之後Find的複雜度,使DSU操作的時間複雜度答案近$O(\alpha{N})$的程度。\\
	(可以想想為什麼兩集合rank相同時,合併後要$+1$?

	\end{document}
