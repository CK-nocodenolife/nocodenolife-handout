\documentclass[main.tex]{subfiles}
\begin{document}

\chapter{pbds}
\section{前言}
這份講義要介紹的是一個GNU C++中的函式庫--pbds(policy-based data structure),俗稱平板電視。他跟STL有一點類似,但是他其中的某些功能是STL無法達到的。大家應該都知道STL在競賽中的重要性,因此pbds也絕對是幫助你邁向程式設計競賽成功的好幫手!話不多說,我們就趕快來看看怎麼用pbds吧!
\section{pbds介紹}
接下來就向各位介紹pbds的用法及其中的資料結構。\\
首先,無論是使用pbds中的人和資料結構,都需要打上這兩行:\\
\begin{C++}
#include <ext/pb_ds/detail/standard_policies.hpp>
using namespace __gnu_pbds;//there are two _ before gnu, but only one _ between gnu and pbds
\end{C++}

以下就開始正式開始介紹pbds中的資料結構囉!
\subsection{Hash Table}
就跟STL中的\inline{unordered\_map}幾乎一樣(可用中括號隨機存取、\inline{insert、delete
、clear、find、查看size}、自定hash function,但不能\inline{count}),它分成\inline{gp\_hash\_table}和 \inline{cc\_hash\_table},使用並無差異,只是處理碰撞的方式不同。\\
而使用時還要加上:\inline{\#include<ext/pb\_ds/assoc\_container.hpp>}。
讓我們直接來看一下範例程式碼吧!\\
\begin{C++}
#include<bits/stdc++.h>
#include<ext/pb_ds/detail/standard_policies.hpp>
#include<ext/pb_ds/assoc_container.hpp>
using namespace std;
using namespace __gnu_pbds;
int main()
{
  gp_hash_table<int,int> r;
  for(int i = 0;i < 5000000;i++)
  {
    r.insert({i,i+1});
  }
  cout << (r.find(7))->second << endl;
  for(int i = 0;i < 5000000;i++)
  {
    r.erase(i);
  }
  if(r.find(7) == r.end())
  {
    cout << "No!" << endl;
  }
}
\end{C++}

之後我又將\inline{cc\_hash\_table}換成\inline{gp\_hash\_table}和\inline{unordered\_map}。跑一樣的程式碼,\inline{cc\_hash\_table}跑2.42秒左右,\inline{gp\_hash\_table}跑2.53秒左右,\inline{unordered\_map}則跑了5.06秒。因此若覺得\inline{unordered\_map}常數太大的人可以考慮用用看pbds的hash table喔!

\subsection{Priority Queue}
對,這也跟STL的\inline{priority\_queue}差不多,STL的\inline{priority\_queue}支援的操作它都支援,而為了防止與STL中的\inline{priority\_queue}搞混,在使用上要特別注意在宣告的時候要加上\inline{\_\_gnu\_pbds::},以確保用到的是pbds中的\inline{priority\_queue}。宣告方式也有些微差異,可在底下範例中看到。相較於STL的\inline{priority\_queue}只有一種,pbds的\inline{priority\_queue}可以在宣告時加上一個tag,以決定\inline{priority\_queue}的實作方式。使用不同的實作方式也就會導致不同操作有不同複雜度,簡單整理如下:\\

\begin{tabular}{|m{4cm}|m{2cm}|m{2cm}|m{2cm}|m{2cm}|m{2cm}|}
\hline
 & \inline{push} & \inline{pop} & \inline{modify} & \inline{erase} & \inline{join} \\
\hline
\inline{std::priority\_queue} & 最差 $\Theta(n)$ & 最差$\Theta(\log n)$ & 最差$\Theta(n\log n)$ & $\Theta(n\log n)$ & $\Theta(n\log n)$ \\
 & 均攤 $\Theta(\log n)$ & & & &\\
\hline
\end{tabular}

其中的pairing heap和binomial heap較常用到。\\
或許會有人好奇,要怎麼修改\inline{priority\_queue}內部的元素呢?\\
首先,pbds的\inline{priority\_queue}的iterator宣告如下:
\begin{C++}
__gnu_pbds::priority_queue<int>::point_iterator it;
\end{C++}
此外,將元素\inline{push}到pbds的\inline{priority\_queue}的時候會有回傳值,即為該元素所在的iterator,只要在\inline{push}的時候紀錄一下,就可以利用\inline{modify}函式和iterator來進行修改囉!\\
使用時別忘了加\inline{\#include<ext/pb\_ds/priority\_queue.hpp>}。\\
給個範例程式:
\begin{C++}
#include<bits/stdc++.h>
#include<ext/pb_ds/detail/standard_policies.hpp>
#include<ext/pb_ds/priority_queue.hpp>
using namespace std;
using namespace __gnu_pbds;
int main()
{
  __gnu_pbds::priority_queue<int,less<int>,binomial_heap_tag> pq;//型態 比較函式 tag
  __gnu_pbds::priority_queue<int,less<int>,binomial_heap_tag>::point_iterator it[20];
  for(int i = 0;i < 10;i++)
  {
    it[i] = pq.push(i);
  }
  pq.modify(it[0],7122);
  cout << pq.top() <<endl;
}
\end{C++}
有沒有覺得寫dijkstra的時候會派上用場呢?

接著示範合併(join),注意被合併的\inline{priority\_queue}的大小會歸零。\\
\begin{C++}
#include<bits/stdc++.h>
#include<ext/pb_ds/detail/standard_policies.hpp>
#include<ext/pb_ds/priority_queue.hpp>
using namespace std;
using namespace __gnu_pbds;
int main()
{
  __gnu_pbds::priority_queue<int,less<int>,pairing_heap_tag> a,b;
  for(int i = 0;i < 5;i++)
  {
    a.push(i);
    b.push(i+5);
  }
  a.join(b);
  cout << a.size() << " " << b.size() <<endl;
  cout << a.top() <<endl;
}

\end{C++}
\subsection{\inline{tree}}
接下來講的就是大家常常抱怨的\inline{set}和\inline{map}的進化版了啊,就是pbds中的平衡二元樹。它除了\inline{insert、erase、lower\_bound、upper\_bound}等\inline{set}已經有的操作之外,還可以找$k$小值(\inline{find\_by\_order},回傳iterator(0-based))和求排名(\inline{order\_of\_key},回傳一整數(0-based)),還支援類似merge-split treap的\inline{join}(\inline{join}的時候兩棵樹的類型要相同,沒重複元素,且key值要左小右大)還有\inline{split}。\\
它的宣告長這樣:\\
\begin{C++}
tree<int,null_type,less<int>,rb_tree_tag,tree_order_statistics_node_update> tr;
\end{C++}

前三格就是STL \inline{map}中的key、對應的資料型態,和比較函式。第四格是各種樹的tag,有\inline{rb\_tree\_tag、splay\_tree\_tag、ov\_tree\_tag,其中的rb\_tree\_tag}較為常用。其他兩個的用處我不是很了解,但它在一般使用上的常數較高,盡量避免使用。最後就是更新方式了,不知道要打甚麼就照抄吧!\\
它其實還有一些更進階的用法,像是自行定義點的更新方式,不過我也不會,而且蛋餅說還不如自己刻一顆treap,所以這邊我就不多說大家有興趣的自行研究喔!
也記得要加上\inline{\#include<ext/pb\_ds/assoc\_container.hpp>}。\\
嗯,一樣來看範例程式。\\
\begin{C++}
#include<iostream>
#include<ext/pb_ds/assoc_container.hpp>
#include<ext/pb_ds/tree_policy.hpp>
using namespace std;
using namespace __gnu_pbds;
int main()
{
  tree<int,null_type,less<int>,rb_tree_tag,tree_order_statistics_node_update> a,b;
  for(int i = 0;i < 10;i++)
  {
    a.insert(i*2);
  }
  a.erase(18); // tr = {0,2,4,6,8,10,12,14,16}
  cout << *a.lower_bound(11) << endl; // 原本set和map支援的操作都有
  cout << *a.find_by_order(0) << endl; 
  cout << *a.find_by_order(5) << endl;
  cout << a.order_of_key(9) << endl;
  cout << a.order_of_key(10) << endl;
  a.split(6,b);
  for(auto i : a)
  {
    cout << i << " ";
  }
  cout << endl;
  for(auto i : b)
  {
    cout << i << " ";
  }
  cout << endl;
  a.join(b);
  cout << a.size() << endl;
}
\end{C++}
\subsection{Trie}
大家可能會覺得剛剛的資料結構或許很厲害,但其實都只是STL之中不同資料結構的變形。但接下來要介紹的資料結構,STL之中就沒有了呢!它,就是大名鼎鼎的Trie!

首先先介紹宣告方式,它的宣告方式比較複雜,我盡量想辦法簡單一點。\\
它完整的宣告長這樣:
\begin{C++}
trie<string,null_type, trie_string_access_traits<>, pat_trie_tag, trie_prefix_search_node_update> tr;
\end{C++}
第一個參數一定要是字串型態,第二格可以是其他的資料型態(類似\inline{map}的感覺),不用的話就用\inline{null\_type},第三格是比較函式,第四格是trie的種類,最後則是更新方式。一般使用時後三格可以照打,有興趣者可自行研究。
那你可以對它做啥操作呢?\\
首先就是最基本的\inline{insert、delete、find字串,也可以進行合併(join)和分割(split)(要注意的是,join}的兩個trie一定要第一個字典序比較小,第二個比較大)。它還有一個很特別的操作,就是\inline{prefix\_range}的這個操作,它會回傳一個pair,代表以某個前綴開頭的字串的開頭的iterator和結尾的iterator。這樣講不太清楚,不如直接看範例吧!\\
要記得加上\inline{\#include<ext/pb\_ds/assoc\_container.hpp>}。\\
\begin{C++}
#include <iostream>
#include <ext/pb_ds/assoc_container.hpp>
#include <ext/pb_ds/detail/standard_policies.hpp>
using namespace std;
using namespace __gnu_pbds;
int main()
{
  trie<string, null_type, trie_string_access_traits<>,pat_trie_tag,trie_prefix_search_node_update> a,b,c;
  a.insert("her");
  a.insert("hers");
  b.insert("she");
  b.insert("his");
  a.join(b);
  auto temp = a.prefix_range("h");
  for(auto i = temp.first;i != temp.second;i++)
  {
    cout << *i << " ";
  }
  cout <<endl;
  a.delete("hers");
  a.split("r",b);
  cout << a.size() << " " << b.size() <<endl;
}
\end{C++}
\section{習題}
其實pbds只是一個方便使用的工具,裡面很多的資料結構都可以自己手刻出來,因此我也不知道要放甚麼習題呢!大家好好思考下面的題目吧!
\problem{忍者調度問題 (TIOJ4544 1429)}{
	給定一個$N$個節點的有根樹,每個節點有兩個值($C$和$L$),再給一個整數$M$,你要找到任意個節點,使的他們$C$值的和小於$M$,且選定的節點數乘上所有選定節點的某個共同祖先的$L$值最大,輸出該最大值。($N   \leq10^5$,$M  \leq 10^9$)
}

\end{document}
