\documentclass[main.tex]{subfiles}
\begin{document}

\chapter{最短路徑 (Shortest Path)}
\problem{不定向邊(TIOJ 1290)}{
給一個圖,單向邊,邊有cost,一開始邊是未決定方向的,你可以對任一條邊決定一個方向。
接著,問起點到終點最短路徑長。如果到不了的話請輸出\texttt{"He is very hot"(不含雙引號)}。\\
輸入包含多組測試資料。 \\
$1\leq|V|\leq1000, 1\leq|E|\leq|V|^2$
}
給定一張圖,邊上面有邊權,\textbf{最短路徑}是由起點到終點、經過邊權的和最小的路徑,可能有許多條,也可能不存在。 最短路徑不見得是邊最少、點最少的路徑。 以下將介紹數種最短路徑的演算法。
\hint{若圖上有負環,最短路徑無法定義;而求取最短簡單路徑(不經過重複點的路徑)也會變為\texttt{NP-complete}問題,因為一個無權圖的哈密頓路徑可以在多項式時間歸約成每條邊權都是$-1$的最短簡單路徑。(可以想一想!)}
\section{BFS}
奇怪,不是感覺之前在圖上找最短路徑就是直接用BFS \ $O(|V|+|E|)$解決嗎?不行不行!!要注意的是,BFS只能套用在邊權都等於一的情況下,而普通的圖邊權不一定為一,BFS就拿它沒轍了。不過,其實待會介紹的各種演算法有些也跟BFS的概念有些相近喔!
\section{Relaxation}
先介紹一個接下來每個演算法都會用到的概念--鬆弛。若目前找到的路徑上有一部分可以用其他更短的路徑代替,我們當然可以用較短的那段代替較長的路徑,整個路徑也會因此而變得更短。我們把這個動作叫做\textbf{鬆弛(Relax)}。

\begin{center}
\begin{tikzpicture}[
roundnode/.style={circle, draw=green!60, fill=green!5, very thick, minimum size=7mm},
keynode/.style={circle, draw=black!60, very thick, minimum size=7mm}
]
%Nodes
\node[roundnode] (start)                    	 {\texttt{s}};
\node[roundnode] (marked) [below right=of start] {\texttt{k}};
\node[roundnode] (finish) [above right=of marked]{\texttt{t}};

%Edges
\path[->,draw,thick]
	(start) edge (finish)
;
\path[->,draw,thick,red]
	(start) edge (marked)
	(marked) edge (finish)
;
\end{tikzpicture}
\end{center}
\texttt{如上圖所示,由s經過k到t的路徑可以用直接從s到t的路徑代替。}

\section{單點源最短路徑}
單點源最短路徑問題,即是求出一個圖當中,一個固定的起點到各點的最短路徑。
\subsection{Bellman-Ford}
直接依照鬆弛的定義來鬆弛看看吧!暴力地每一回合將所有邊都鬆弛一次!總共做$|V|-1$次,時間複雜度為$O(|V||E|)$。

為什麼是$|V|-1$次呢?因為圖上沒有負環,所以最短路徑一定是簡單路徑,否則可以在經過同一點時從有環的路徑上移除環(一定是正環,路徑會變短),而這條路徑最多經過$|V|$個點,所以經過的邊數最多是$|V|-1$;***每次鬆弛所有邊實際上是嘗試用比前一輪多一條的邊代替最短路徑,有點類似BFS,第n次鬆弛操作保證了所有使用n個邊的路徑最短,***因此鬆弛$|V|-1$回合後,就可以找到從起點到所有點的最短距離。這同時也告訴我們如果在第$|V|$回合仍有邊可以鬆弛,則代表圖上有負環。

大家或許覺得這個複雜度有點差,但這個演算法可以處理有負邊權(無負環)的最短簡單路徑,而且還可以找出圖中有沒有負環,所以不要小看它喔! 
\begin{C++}
#define ff first // 小寫字母教 OwO
#define ss second
typedef pair<int,int> pii;
vector<pii> g[N+1]; // adjacency list
int dis[N+1], v, relax;
void BellmanFord(int src){
	dis[src]=0;
	// 重複 Relax V-1次,第V次仍有鬆弛則表示有負環
	for(int r = 0;r < v;r++){
		relax = false;
		for(int i = 0;i < v;i++) // O(E)
			for(auto e:g[i])
				if(dis[e.ss]>dis[i]+e.ff) // Relax
					dis[e.ss]=dis[i]+e.ff, relax=true;
		if(!relax) return;
	}
	// 有負環
}
\end{C++}

\subsection{SPFA - 猜猜它的全名}
這個算法其實就是剛剛那個大家覺得很暴力的Bellman-Ford演算法的改進,他就像從皮丘進化成了皮卡丘,電度上升了不只一倍。其實它也沒有那麼難啦!他的做法就是從Bellman-Ford的每次鬆弛所有邊,變成每次只鬆弛特定的邊。需要鬆弛的邊只有前一輪更新過端點的邊,我們可以用一個\texttt{queue}來記錄哪些節點旁邊的邊可以拿來鬆弛其他節點。

SPFA有許多奇奇怪怪的優化方式,但在這邊先不贅述太多;有一個優化就是在要把點塞入\texttt{queue}裡面的時候,先看它是否已經在\texttt{queue}裡面,如果該點已經在\texttt{queue}裡面的話,就不用再把那個點放進去一次囉!
\begin{C++}
vector<pii> g[N+1]; // adjacency list
int dis[N+1], inQ[N+1], v, cur;
void SPFA(int src){
	queue<int> q;
	fill(dis,dis+v,INF), fill(inQ,inQ+v,0);
	q.push(src), dis[src] = 0, inQ[src] = 1;
	while(!q.empty()){ // Repeat Relaxing
		cur = q.front(), q.pop(), inQ[cur] = 0;
		for(auto e:g[cur])
			if(dis[e.ss] > dis[cur]+e.ff){
				dis[e.ss] = dis[cur]+e.ff;
				if(!inQ[e.ss])
					inQ[e.ss]=1, q.push(e.ss);
			}
	}
}
\end{C++}
那SPFA的複雜度到底是多少呢 ? 不瞞各位所說,SPFA的平均複雜度還沒有被提出嚴謹的證明,而它的最壞複雜度亦為$O(|V||E|)$,與Bellman-Ford並無二致,不過在隨機圖的試驗中,它的平均運行時間是$2|E|$左右。意想不到的快,對吧?

\subsection{Dijkstra}
\hint{Dijkstra有人念Dijkstra,有人則念Dijkstra,而且還有念Dijkstra的人。}
我們可以把BFS的搜尋順序想成是一層一層向外擴張的同心圓,更新距離為$1$、距離為$2$ $\cdots$距離為$k$的點。Dijkstra可以想成是加上權重的BFS,由小到大以不同的半徑更新節點(半徑最小的同心圓),每次找到距起點最近的未拜訪節點,並嘗試更新其周圍的節點。

從另一觀點看,若圖上沒有負邊,最短路徑的前綴一定是最短路徑,因此每次選擇一最近節點加入\textbf{最短路徑樹},並嘗試鬆弛周遭節點,是一種 Greedy 算法,和 Prim 演算法(某種最小生成樹演算法)有異曲同工之妙。

什麼是最短路徑樹呢?就是把從起點到每一個點的最短路徑畫出來,形成的一棵以起點為根的樹。因為起點到每個點的最短簡單路徑都只有一條,若有兩條以上就可以把一條拔掉,所以最短路徑們會形成一棵樹。

要記得注意的是,這個演算法只能套用在沒有負邊權的情況下,否則可能會發生以下這種狀況:

\begin{center}
\begin{tikzpicture}[
roundnode/.style={circle, draw=green!60, fill=green!5, very thick, minimum size=7mm},
keynode/.style={circle, draw=black!60, very thick, minimum size=7mm}
]
%Nodes
\node[roundnode]      (start)						        {\texttt{s}};
\node[roundnode]      (marked)  	 [above right=of start] {\texttt{k}};
\node[roundnode]      (finish)       [below right=of marked] {\texttt{t}};
 
%Edges
\path[->,draw,thick,red]
    (start) edge node[above left] {$5$} (marked)
    (marked) edge node[above right] {$-4$} (finish)
;
\path[->,draw,thick,blue]
    (start) edge node[below] {$3$} (finish)
;

\end{tikzpicture}
\end{center}

\texttt{假設s是起點,t是終點,一般的Dijkstra算法會先入為主地認為直接走到t比較快,但其實先走到看似比較遠的k,到最後可能會有比較短的路徑!}

\begin{C++}
vector<pii> graph[N+1];// adjacency list
int dis[N+1] = {}, v, e;
void Dijkstra(int src){
	//期望節點清單,依距起點距離排序 {dist,vertex}
	priority_queue<pii,vector<pii>,greater<pii> > pq;
	pii cur;
	fill(dis,dis+v,INF);
	dis[src] = 0;
	pq.push({0,src});
	//每個節點只會被更新一次
	for(int i = 0;i < v;i++){
		//將已更新的節點從清單移除(Lazy Deletion)
		do cur = pq.top(), pq.pop();
		while(cur.ff>dis[cur.ss]);
		for(auto e:graph[cur.ss])// Relax
			if(dis[e.ss] > cur.ff+e.ff)
				dis[e.ss] = cur.ff+e.ff,
			pq.push({dis[e.ss],e.ss});
	}
}
\end{C++}
利用STL的\texttt{priority\_queue}實作找出和起點距離最小節點的部分,時間複雜度$O(|E|\log|V|)$。注意中間有Lazy Deletion的技巧,Dijkstra中已經更新過的節點必須從heap裡面移除,不過\texttt{priority\_queue}不支援刪除任意數字的操作,因此我們在pop出來的時候才跳過這些節點。如果用費波納契堆實作可以達到$O(|E|+|V|\log|V|)$的時間複雜度,但幾乎不太可能手刻出來。

\section{全點對最短路徑}
有時候我們會想要求每一個點到每一個點間的最短距離,我們當然可以對每個頂點都做一次Bellman-Ford或Dijkstra,但這樣複雜度分別是$O(|V|^2|E|)$和$O(|E||V|\log|V|)$,還有優化的空間。
\subsection{Floyd Warshall}
Floyd Warshall是利用DP的想法,只需要$O(|V|^3)$即能求出所有點對之間的距離。DP式大概長這樣: 
\begin{center} $dp[i][j][1] = w[i][j]$ \end{center}
\begin{center} $dp[i][j][k] = \text{min}(dp[i][j][k-1], dp[i][k][k-1]+dp[k][j][k-1])$ \end{center}
其中$dp[i][j][k]$表示從$i$到$j$,且中途只用到前$k$個點當中繼點的最短距離,也就是說每次看$k$能不能替換$i$到$j$的路徑中一部份,逐步開放第$1,2\cdots k$個點。順帶一提用鄰接矩陣就可以直接枚舉$i,j$就好,而且照著$k$遞增的順序dp可以滾動。Code超短,常數也超低,所以如果測資很鬆可以偶爾拿來做單點源最短路徑。
\begin{C++}
// adjacency matrix
int dis[N+1][N+1], v;
void FloydWarshall{
	for(int k = 0;k < v;k++)
		for(int i = 0;i < v;i++)
			for(int j = 0;j < v;j++)
				if(dis[i][j] > dis[i][k]+dis[k][j])
					dis[i][j] = dis[i][k]+dis[k][j];
}
\end{C++}

\section*{習題們}
習題的唷
\problem{乳酪的誘惑(ZJ d273)}{給定$n\times m$方格上的障礙物,求起點走到終點的最短路徑長。$n,m\leq1000$。}
\problem{地道問題(TIOJ 1509)}{給定一張有向圖,問你編號一的點到所有其他點的最短距離和加上所有點到編號一的點的最短距離和。$(|V|,|E|\leq10^6)$}
\problem{貨物運送計畫(TIOJ 1641)}{邊權乘積最小路徑$(|V|\leq10^4,|E|\leq2\times10^2)$。}
\problem{阿思歐思(TIOJ 1685)}{路徑上點權max最小路徑$(|V|\leq200,|E|\leq10^4)$最多有$10^4$次詢問。}
\problem{Wormholes(UVa 558)}{判斷一張有向圖中是否有負環。$(|V|\leq1000,|E|\leq2000)$}

\problem{Going in Cycle!!(UVa 11090)}{給定一張有向圖,找出一個環,使得環上邊權的平均最小。$(|V|\leq50,\text{邊權}\leq10^7)$}
\problem{圖論專家(ZJ d243)}{給定一張無向圖,求起點到終點的嚴格第k短路徑長。}
\problem{成為神奇寶貝大師!(競程日記 MCC \#6 pE)}{\href{https://oj.icpc.tw/c/36/E}{\underline{\texttt{https://oj.icpc.tw/c/36/E}}} (題敘略)}

\end{document}
