\documentclass[main.tex]{subfiles}
\begin{document}

\chapter{基礎計算幾何}

計算幾何的焦點與一般數學的幾何題不大相同,主要會圍繞著向量以及解析幾何(有架座標的幾何)進行。在高中的競賽(如 TOI / IOI )中,雖然幾何的題目不常出現,但每次出現往往伴隨著不小的實作量以及拆解包裝的過程。本講義將會以高中的競賽為主,著重於二維平面的計算幾何以及其實作。

本章節將會介紹如何實作基礎的計算幾何元件,以及一些經典的計算幾何習題。

\subsection{先備知識與技能}

\begin{itemize}
	\item 了解三角函數的定義
	\item 基礎的向量操作(會用紙筆算)
	\item 基礎的 STL 容器與函式使用
	\item 一顆相信自己的心(你一定可以實作出來的!)
\end{itemize}

\subsection{開始前的碎碎念}

\begin{itemize}
	\item 這份講義的數學部份並不嚴謹,例如會混用 $A$ 表示平面上點 $A$ 以及向量 $\vec{OA}$ 。
	\item 「內部」、「上」的定義若未特別說明,皆是非嚴格的(在邊界或端點上也算)。
	\item 計算幾何的實作流派非常多,建議多多參考高手的 submission 以及 codebook ,往往可以學到許多特殊的技巧。本講義也會盡量提及所有實作方式。
	\item 相較於其他主題,計算幾何的內容較為繁雜,請搭配實作練習幫助吸收,不要一次吃到撐(?)
	\item 練習建議:筆者自己在學習的時候,每一題都會重新打一遍需要的模板。推薦每次不要寫太多題(只寫一題也可以),但請多練習憑記憶打出模板,直到你認為在比賽等高壓環境下也能毫無錯誤的產出為止再使用複製貼上。
	\item 記得好好看測資限制(三點不共線等),往往可以發現更簡單的作法或是關鍵的觀察。
	\item 不要科技中毒!好好利用性質簡化問題吧!
\end{itemize}

\section{處理浮點數誤差}
浮點數誤差麻煩又可怕,運算時常常可以打破基本的數學規則\footnotemark。然而在計算幾何中,不論多努力避免,還有必須使用浮點數的時候,例如:歐氏距離、線段交點。這些誤差會在我們做「點是否在線上」等二元判斷的時候造成很大的問題,因此,我們會需要使用一些工具來處理誤差。廢話不多說,直接上 code :

\begin{C++}
const double eps = 1e-8;
int sign(double a) { return fabs(a) < eps ? 0 : (a > 0) - (a < 0); }
\end{C++}

從程式碼中不難理解 \inline{eps} ( epsilon 的縮寫,常以 $\varepsilon$ 表示)代表誤差的容許範圍,也就是「多小以下的數字可以忽略」,而 \inline{sign} (數學中常以 sgn 或 signum 表示)則是回傳考慮了誤差後,一個浮點數應該是正數、負數還是 0 。
$\varepsilon$ 的大小通常是 $10^{-8} \sim 10^{-10}$ 之間的數,大小基本上是個經驗法則,賽中如果出錯了也會調整 $\varepsilon$ 看看是不是誤差導致的。如果想要知道怎麼好好估計 $\varepsilon$ 的大小,請參考進階計算幾何講義。

\footnotetext{不信你可以試試看 \inline{0.1 + 0.2 == 0.3} ,答案是 \inline{false} 。}

\section{向量}

眾所周知, \inline{C++} 裡面的 \inline{vector} 不是向量\footnotemark。為了能讓程式進行基本的向量運算,我們需要自行實作向量型別。以下會以筆者熟悉的風格撰寫。

\footnotetext{也有人說 \inline{std::array} 和 \inline{std::vector} 應該要互換名字比較正確。}

一樣直接上程式碼:

\begin{C++}
// 向量本體
struct Vec {
	double x, y;
	Vec(double x = 0, double y = 0): x(x), y(y) {}
};

// 加減乘除
Vec operator + (Vec a, Vec b)
{ return Vec(a.x + b.x, a.y + b.y); }
Vec operator - (Vec a, Vec b)
{ return Vec(a.x - b.x, a.y - b.y); }
Vec operator * (Vec a, double b)
{ return Vec(a.x * b, a.y * b); }
Vec operator / (Vec a, double b)
{ return Vec(a.x / b, a.y / b); }

// 內外積
double dot(Vec a, Vec b)
{ return a.x * b.x + a.y * b.y; }
double cross(Vec a, Vec b)
{ return a.x * b.y + a.y * b.x; }

// 向量長
double abs2(Vec a)
{ return dot(a, a); }
double abs(Vec a)
{ return sqrt(dot(a, a)); }

// 可有可無的輸入輸出, debug 時好用
istream& operator >> (istream& is, Vec v)
{ return is >> v.x >> v.y; }
ostream& operator << (ostream& os, Vec v)
{ return os << '(' << v.x << ',' << v.y << ')'; }
\end{C++}

其實 \inline{+ * /} 很少用,通常只會打 \inline{-} 而已。善用複製貼上會事半功倍喔!

而在整數向量底下, \inline{abs2} 也會是整數,因此要取最常的向量的長度時不妨先以 \inline{abs2} 比較,最後再開根號。

\eeric{常見的向量實作方式還有使用 \inline{pair<double, double>} 或是 \inline{std::complex<double>},後者可以省去加減乘除以及輸入輸出的實作,內外積也較簡短,但是會把 x 和 y 鎖死,不能作為變數名稱。各個流派各自有優缺點,建議多看看各家實作再決定一個喜歡的。}

\subsection{內外積的應用}

根據數學上的定義,我們知道 $\vec{A} \cdot \vec{B} = |\vec{A}||\vec{B}| \cos \theta, \vec{A} \times \vec{B} = |\vec{A}||\vec{B}| \sin \theta$ 。因為向量長度(通常)恆正,因此內外積的正負號就可以來判斷兩向量的角度關係:

% 記得加圖片
\begin{table}[h]
	\centering
	\begin{tabular}{ c | c c c }
		輸出 & 1 & 0 & -1 \\ \hline
		\inline{sign(dot(A, B))} & 同向 & 正交 & 反向 \\
		& $|\theta| < 90^\circ$ & $|\theta| = 90^\circ$ & $|\theta > 90$ \\ \hline
		\inline{sign(cross(A, B))} & 逆時針 & 共線  & 順時針 \\
		& $0^\circ < \theta < 180^\circ$ & $\theta = 0^\circ \cup 180 ^\circ$ & $-180^\circ < \theta < 0\theta$ \\ \hline
	\end{tabular}
	\caption{內外積正負號與向量夾角的關係}
	% \label{tab:<+label+>}
\end{table}

除此之外 $\vec{A} \times \vec{B}$ 也等於兩向量所張平行四邊形的有向面積,同時也是三角形 $\triangle OAB$ 有向面積。

為了實作上的方便,通常我們會實作兩個函式,判斷三點 $A, B, C$ 所成向量 $\vec{AB}, \vec{AC}$ 的角度關係:

\begin{C++}
int ori(Vec a, Vec b, Vec c)
{ return sign(cross(b - a, c - a)); }
int dir(Vec a, Vec b, Vec c)
{ return sign(dot(b - a, c - a)); }
\end{C++}

\section{線段、射線與直線}

兩點可以決定唯一的線段、射線與直線,程式碼也是如此實作:

\begin{C++}
typedef pair<Vec, Vec> Seg;
\end{C++}

注意,接下來提到線段的時候都是指包含兩端點的線段。

\subsection{點到直線的距離}

雖然到直線 $ax + by + c = 0$ 的距離有公式可以帶,但是這裡有個簡單的方式:剛剛有提到兩向量外積大小就是其所張平行四邊形面積,而平行四邊形面積又等於底乘高,假設我們要求 $C$ 到 $\overleftrightarrow{AB}$ 的距離,那就讓 $\overline{AB}$ 當底,高就是所求距離!

$$dis(C, \overleftrightarrow{AB}) = \frac{|\vec{AB} \times \vec{AC}|}{|\vec{AB}|}$$

\begin{table}[h!]
	\begin{tabular}{p{7cm} p{7cm}}
		\begin{center}
			\includegraphics*{images/Geometry/Basic/PointToLine1.pdf}
		\end{center} &
		\begin{center}
			\includegraphics*{images/Geometry/Basic/PointToLine2.pdf}
		\end{center}
	\end{tabular}
\end{table}

\begin{C++}
double PointToLine (Seg s, Vec p) {
	Vec v = s.second - s.first;
	return fabs(cross(v, p)) / abs(v);
}
\end{C++}

\subsection{點到線段的距離}

比直線稍微麻煩一點,因為有些時候點到線段沒辦法做垂線,最短距離會是到線段端點!至於什麼情況不能畫垂線呢?畫畫看就知道了:
\begin{center}
	\includegraphics*{images/Geometry/Basic/PointToSeg.pdf}
\end{center}

可以發現,如果 $\angle CAB$ 和 $\angle CBA$ 其中一個超過 $90 ^ \circ$ ,那麼就不能畫垂線了。那要怎麼判斷三個點的夾角跟 $90 ^ \circ$ 的大小關係呢?內積!

\begin{C++}
double PointToSeg(Seg s, Vec p) {
	if (dir(s.first, s.second, p) < 0 or
			dir(s.second, s.first, p) < 0)
		return min(abs(p - s.first), abs(p - s.second));
	Vec v = s.second - s.first;
	return fabs(cross(v, p)) / abs(v);
}
\end{C++}

\subsection{判斷點是否在直線上}

其實就是判斷三點共線,方法是用所張有向面積為 $0$ 來判斷。

實作時通常不會包成 \inline{Seg} 的資料型態,因此下面的實作以 \inline{a, b} 來取代 \inline{s.first, s.second} 。

\begin{C++}
bool PointOnLine(Vec a, Vec b, Vec c)
{ return ori(a, b, c) == 0; }
\end{C++}

\subsection{判斷點是否在線段上}

再多判斷 $\vec{AC}$ 和 $\vec{BC}$ 是否非同向即可!

\begin{C++}
bool PointOnSeg(Vec a, Vec b, Vec c)
{ return ori(a, b, c) == 0 and dir(c, a, b) <= 0; }
\end{C++}

\subsection{判斷線段與直線是否相交}

假設線段是 $\overline{AB}$ ,直線是 $\overleftrightarrow{CD}$ ,那只要 $A, B$ 兩個有至少一個在 $\overleftrightarrow{CD}$ 上、或是兩個在 $\overleftrightarrow{CD}$ 的不同側,就代表有相交!

\begin{C++}
bool SegLineInter(Vec a, Vec b, Vec c, Vec d) {
	return PointOnLine(c, d, a) or PointOnLine(c, d, b)
		or ori(c, d, a) != ori(c, d, b);
}
\end{C++}

\subsection{判斷兩線段是否相交}

這邊有很多種情況要判斷:

\begin{center}
	\includegraphics*{images/Geometry/Basic/SegInter.pdf}
\end{center}

不難發現可以分為兩種情況:

\begin{enumerate}
	\item 其中一線段的端點在另一線段上
	\item $\overline{AB}$ 和 $\overleftrightarrow{CD}$ 相交,$\overline{CD}$ 和 $\overleftrightarrow{AB}$ 也相交。
\end{enumerate}

\begin{C++}
bool SegInter(Vec a, Vec b, Vec c, Vec d) {
	return PointOnLine(a, b, c) or PointOnLine(a, b, d)
		or PointOnLine(c, d, a) or PointOnLine(c, d, b)
		or (ori(a, b, c) != ori(a, b, d)
				and ori(c, d, a) != ori(c, d, b));
}
\end{C++}

\subsection{求兩線段交點}

這裡會用到一些國中的幾何:我們可以知道 $\overline{CX} : \overline{DX} = d_1 : d_2 = \triangle ACB : \triangle ADB$ ,而三角形面積我們可以用外積求!不過,要特別注意外積得到的面積會是有向面積,在計算時要注意正負號喔。在使用前,記得保證兩線段不平行喔。

\begin{center}
	\includegraphics*{images/Geometry/Basic/SegInterPoint.pdf}
\end{center}

\begin{C++}
Vec SegInterPoint(Vec a, Vec b, Vec c, Vec d) {
	assert(cross(b - a, d - c) != 0);
	double acb = cross(b - a, c - a);
	double adb = cross(d - a, b - a);
	return (c * adb + d * acb) / (adb + acb);
}
\end{C++}


另外,這個方法其實可以用在直線相交,也就是兩線段沒有相交的話會得到它們無限延伸後的交點,不過正確性就交給讀者證明囉。

\section{向量基礎操作的基礎應用}

既然已經會寫基礎的模板了,就來實際應用它們吧!

\subsection{線段 $\overline{AB}$ 的中垂線段}

說中垂線是騙人的!假設我們要求線段 $\overline{AB}$ 的中垂線段 $\overline{CD}$ ,其實我們只要求出符合以下條件的線段就好了:

\begin{enumerate}
	\item 過 $\overline{AB}$ 的中點 $C = \frac{A + B}{2}$ 。
	\item 和 $\overline{AB}$ 垂直 $\rightarrow \vec{AB} \cdot \vec{CD} = 0$ 。
\end{enumerate}

不難發現,假設 $\vec{AB}$ 是 $(x, y)$ ,那只要令 $\vec{CD} = (-y, x), D = C + \vec{CD}$ 就一定會垂直!

\begin{C++}
Seg SegBisector(Vec a, Vec b) {
	Vec c = (a + b) / 2;
	Vec ab = b - a;
	Vec cd = Vec(-ab.y, ab.x);
	return Seg(c, c + cd);
}
\end{C++}

\subsection{角 $\angle BAC$ 的角平分線段}

說角平分線段也是騙人的!

\end{document}
