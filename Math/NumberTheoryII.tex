\chapter{數論 II}
\section*{前言}
	之前大家學過了一些數論,這次再繼續精進我們的數論技巧吧!這裡會探討比較進階的話題,像是各種數論函數、反演(不是幾何的!)、轉換等。讓我們來進入這個神奇的世界吧!
\section*{數論函數}
	\subsection*{何謂數論函數}
	顧名思義,數論函數就是「數論常常探討、用到的函數」!數論函數通常為$\mathbb{Z} \rightarrow \mathbb{Z}$(整數打到整數)的數字,所以如果沒有註明,就滿足之。這個章節所指的「數」指「整數」。我們主要想要探討的是「乘法函數」:
	\definition{乘法函數}{
		一個函數$f(x)$為一個\textbf{乘法函數},若其滿足對於兩個互質的數字$a$、$b$,
		\begin{equation*}
			f(ab) = f(a)f(b)
		\end{equation*}
		若$a$、$b$不需要互質,則其被稱為一個\textbf{完全乘法函數}。一些乘法函數的例子包括:$f(x) = c$($c$是任意數字)、$f(x) = x$、$f(x) = x^k$($k$是任意數字)等。
	}
	\subsection*{歐拉$\varphi$函數}
		\definition{歐拉$\varphi$函數}{
			對於一個正整數$n$,我們定義
			\begin{equation*}
				\varphi(n) = \#\{x | (n, x) = 1\}
			\end{equation*}
			也就是「小於$n$的正整數中,與$n$互質的個數」。
		}
		\subsubsection{$\varphi$是乘法函數}
		那要如何證明其為乘法函數呢?假設$a, b$互質。則我們將所有少於$ab$的數字分為三個集合:$X, Y, Z$,分別代表與$a$互質、與$b$互質、與$ab$互質。則我們想要找到一個$X \times Y \rightarrow Z$的雙射$f$。令$x \in X$、$y \in Y$,則存在一個數字$t$滿足
		$$\begin{cases}
		t\equiv x \mod a\\
		t\equiv y \mod b
		\end{cases}$$
		則由中國剩餘定理,$t = (Ax + By) \pmod{ab}$,$A = b \cdot \left[b^{-1} \pmod a\right]$, $B = a \cdot \left[a^{-1} \pmod b\right]$。現在假設存在一組$x' \in X$、$y' \in Y$,且$t' = (Ax' + By') = t$。我們想要證明$x' = x$且$y' = y$!
		
		首先,可以知道$Ax + By = Ax' + By'$。故$A(x - x') + B(y- y’) = 0$。兩邊取$a$的餘數:
		\begin{equation*}
			(x - x') + 0 \equiv 0 \pmod a
		\end{equation*}
		故$x \equiv x' \pmod a$,但是因為$x, x'$皆小於$a$,故$x = x'$。同理,$y = y'$。
	\subsubsection{$n$為質數冪}
		不難看出,若$p$為質數,則$\varphi(p) = p - 1$。那如果是質數的冪次呢?在小於$p^k$的數字中,唯有是$p$的倍數的數字不符合。這種數字有$\frac{p^k}{p} = p^{k - 1}$個,扣除則可以得到:
		\begin{equation*}
			\varphi(p^k) = p^k - p^{k - 1}
		\end{equation*}
	\subsubsection{任意數的$\varphi$值}
		有了是乘法函數的性質,且有質數冪了,就可以計算任何數的$\varphi$值了!假設我們要計算$\varphi(n)$值。首先,先來質因數分解:
		\begin{equation*}
			n = \prod_{i = 1}^{\infty} p_i^{\alpha_i}
		\end{equation*}
		此處$p_i$為第$i$個質數,$\alpha_i$為第$i$個質數的次方。則根據以上,
		\begin{equation*}
		\varphi(n) = \prod_{i = 1}^{\infty} \varphi(p_i^{\alpha_i}) = \prod_{i = 1}^{\infty} \left[p_i^{\alpha_i} - p_i^{\alpha_i - 1}\right]
		\end{equation*}
		如果將$n$提出來,則會有更漂亮的公式:
		\begin{equation*}
			\varphi(n) = n \cdot \prod_{p | n} (1 - \frac{1}{p})
		\end{equation*}
		且$p$為質數。
		