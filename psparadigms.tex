\chapter{解題思路}

\section{前言}
	在程式競賽中,有許多的題目是多方面的——搭配了資料結構、演算法、還有許多的巧思:這些巧思或許真的是通靈,但是也有許多是有跡可循的,或者可以讓算法變得更快,要不然能提供有用的思考方法。這些方法是比較常用的,在解題若想不大到可以往這些方面想;也有可能你認為這些概念非常籠統,沒辦法實際應用。沒錯!這些就是比較抽象的思考方向,提供可能的路,而要如何走在於你。
\section{二分搜}
	\subsection{猜數字的例子}
		這是最基礎的思考路徑,很多時候可以將一個演算法的複雜度從線型硬壓成對數複雜度!來看一個簡單的例子,就知道是什麼意思了:
		\problem{Guess My Number(TIOJ 1044)}{
			一開始電腦會產生從$1$到$N$的一個正整數$K$,而你每次可以問一個數字$Q$,而電腦會回傳$K$和$Q$的大小關係;請猜出$K$為何?($1 \leq N \leq 10^{18}$)
		}
		第一個想法一定會是從$1$猜到$N$,一路猜到$N$,一定會中!然而,這樣盲目的猜只能得到殘忍的TLE——$N = 10^{18}$的話,實在跑太久了!而且,沒有充分利用題目給的資訊:大小關係一定很重要!就來想,如果跟你說:「$Q$這個數字太大了!」,代表什麼?代表,對於所有的$X \leq Q$,都不用猜了,一定不會是答案;如果跟你說:「$Q$這個數字太小了」,當然也是一樣,比$Q$小的數字都不用猜了。利用這個資訊,我們可以每次都紀錄目前「答案可能在的區間」$[L, R]$,一開始$[L, R] = [1, N]$,而每次我們都問$M = \lfloor \frac{L +R }{2} \rfloor$,所以區間就可以變成$[L, M - 1]$或$(M + 1, R)$了。顯然地,每次新的區間都會比舊的區間小一半,所以想要區間長度變成$1$(也就是答案只有一個可能)需要$\log_2(N) = O(\log N)$次猜測。所以呀,透過了這樣的「砍半」,可以有效將搜尋從$O(n)$變成非常好的$O(\log N)$!這種技巧稱為\textbf{對答案二分搜}。
	\subsection{更難的例子}
		當然,二分搜的題目沒有都那麼仁慈;來看一個稍微難一點的題目吧!
		\problem{Solve It(UVa 10341)}{
			給定整數$0 \leq p, q \leq 20$和$-20 \leq q, s, t \leq 0$,請解方程
			$$f(x) = pe^{-x} + q\sin(x) + r\cos(x) + s\tan(x) + tx^2 + u = 0$$
			且$0 \leq x \leq 1$。若無解,請輸出\inline{-1}。
		}
		我知道這裡有許多數學好的人,會想要用數學方法直接推解答——別這樣,不需要!只需要知道「一個數字和解答的大小關係」就可以了,而顯然可以$O(1)$(假設運算皆為$O(1)$算出來:
		\begin{C++}
double eps = 1e-9;

double f(double x){
	return (p * exp(x) + q * sin(x) + r * cos(x) + s * tan(x) + t * x * x + u;	
}
		\end{C++}
		也就是是否$|f(x)| < 10^{-9}$,如果是,就說是解了。
		\hint{精度問題:因為電腦存浮點數的方法是近似,所以不能直接判斷是否為$0$,只能判斷是否夠靠近。}
		那要怎麼二分搜呢?就用上次的$L$,$R$吧,一開始答案的值域是$(L, R) = (0, 1)$:
		\begin{C++}
double l = 0, r = 1, m = (l + r) / 2;
while(r - l > eps){
	if(f(m) > 0){
		l = m;	
	} else {
		r = m
	}
	m = (l + r) / 2;
}
		\end{C++}
		而這裡的陷阱就是:$f(x)$在$[0, 1]$上遞減(驗證便得),所以寫的時候要小心。
	\subsection{二分搜什麼時候會用?}
		二分搜也不是萬能的,也會有限制;而知己知彼,才能百戰百勝,所以需要來認識二分搜的短處。首先,如果是這樣的題目:
		\problem{這是微分嗎(經典問題)}{
			電腦生成了一個整數係數二次函式
			$$f(x) = ax^2 + bx + c \; (a > 0)$$
			($a > 0$)每次可以猜一個數字$X$,而我會給你$f(X)$的值。請猜出$f(x)$的最小值發生在哪裡?
		}
		假設最小值發生在$x = K$,而你問的值是$x = Q$,那你可能想要對$f(K) - f(Q)$二分搜到$0$,但是有一個問題:如果在二次曲線的其中一邊,那一定會回傳大於零,但是不知道函式為遞增或遞減所以可能反而一直跑不完!。所以:得到了結論:\textbf{對答案二分搜的時候,被二分搜的變數必須有單調性,或是有明確的分界(也就是說,在一個臨界值之前都不符合,而在一個臨界值之後都符合,找最小符合的答案的時候會需要)}
	\subsection{三分搜}
		那要如何解決上一題呢?這裡要介紹一個相對比較少用(但不代表不重要)的資訊!三分搜不同於二分搜在於它可以處理一個「轉折」,也就是不單調的地方可以有一個。這裡,還是假設目前答案區間在$[L, R]$,則可以猜兩個數字$A = \lfloor \frac{2L + R}{3} \rfloor$,$B = \lfloor \frac{L + 2R}{3} \rfloor$,將區間分為三等份(以後稱為左、中、右)。根據三一律,可以分為三個case:
		\begin{enumerate}
			\item $f(A) > f(B)$,則可以知道$[L, A]$一定沒有最小值,新的答案區間為$[A, R]$
			\item $f(A) = f(B)$,可以知道$[L, A]$,$[B, R]$都不存在最小值,所以新的答案區間為$[A, B]$
			\item $f(A) < f(B)$,同(1)可以知道$[B, R]$一定沒有最小值,新的答案區間為$[L, B]$
		\end{enumerate}
		在最差情況下,每次更新區間,長度都會變成原本的$\frac{2}{3}$,所以複雜度$\log_{\frac{3}{2}}N = O(\log N)$
		\hint{
			不一定要選擇$A = \lfloor \frac{2L + R}{3} \rfloor$,$B = \lfloor \frac{L + 2R}{3} \rfloor$哦!只要是選擇任兩個點在$[L, R]$之間即可。很常用的一個選擇方式是$A = \lfloor \frac{L +R}{2} \rfloor$,$B = \lfloor \frac{L +R}{2} \rfloor + 1$。
		}
	\subsection{STL中的二分搜}
		相信大家都覺得寫二分搜很麻煩吧!幸好,STL也有幫忙二分搜的函數:\inline{lower\_bound}和\inline{upper\_bound}!給定一個\textbf{排序好}的序列,\inline{lower\_bound(iter a, iter b, T t)}回傳在\inline{a}和\inline{b}這兩個指標之間,第一個回傳\inline{大於或等於}\inline{t}的元素。另外一個函式\inline{upper\_bound(iter a, iter b, T t)}回傳在\inline{a}和\inline{b}這兩個指標之間,第一個回傳\inline{大於}\inline{t}的元素。請看以下的範例:
		\begin{C++}
int arr[5] = {4, 5, 13, 71, 92};
cout << *lower_bound(arr, arr + 5, 5) << endl;
cout << *lower_bound(arr, arr + 5, 6) << endl;
cout << *upper_bound(arr, arr + 5, 5) << endl;
cout << *upper_bound(arr, arr + 5, 6) << endl;
		\end{C++}
	輸出:
	\begin{C++}
5
13
13
13
	\end{C++}
	\subsection{習題}
		來練習一下題目吧!
		\problem{Wifi (UVa 11516)}{
			一條街上有編號為$1$到$N$的房子,編號為$i$的房子都和相鄰的房子距離$1$個長度單位。其中$M$個居民(都住在不同房子)想要買最多$K$個發射台(每一個發射台都有固定的發射半徑$R$),使得這$M$個居民的房子都有至少一個接收台在其發射半徑內。請問,如果可以任意放發射台(與房子同位置也可以),最小的半徑$R$為何?
		}
		\problem{Sagheer and Nubian Market(CF 812C)}{
			小沙來到了一個市集,而這個市集賣$n$個東西,編號為$1$到$n$。而每一樣東西都有一個「基本價」$a_i$。但是,這個市集的計價方式很特別:若小沙總共買了$k$樣東西,編號分別為$x_1, x_2, x_3, \dots x_k$,則第$i$項東西的價格是$a_{x_i} + kx_i$。小沙只有$S$元,而他想要買最多樣東西。請輸出小沙最多可以買幾樣東西和買那麼多樣東西所花費的最小價格。($1 \leq n, a_i \leq 10^5$,$1 \leq S \leq 10^9$)
		}

\section{動態規劃 (Dynamic Programming)}
動態規劃(簡稱DP),根據維基百科的定義,是一種通過把原問題分解為相對簡單的子問題的方式求解複雜問題的方法。當我們碰到一個複雜的問題時,有時直接求得此問題的答案是非常耗時或甚至難以完成的,因此我們會想到,或許可以利用原問題的子問題經過一些轉換後得到原問題的答案,而這就是動態規劃最開始的思維。讓我們先以一個大家耳熟能詳的數列來作為例子吧!
\problem{no judge}{
	輸入 $n$,求費氏數列第 $n$ 項。 ($n\leq 10^7$)
}
相信大家都知道什麼是費氏數列,也很輕鬆可以完成這樣一份程式碼:
\begin{C++}
	int f(int n) {
		if (n <= 2) return 1;
		return f(n-1) + f(n-2);
	}
\end{C++}
經過一些數學推導後我們會發現上述程式碼的複雜度為 $O(\varphi^n)$ ,此處$\phi = \frac{1 + \sqrt{5}}{2}$,倘若要計算 $f(10^7)$,我們大約需要經過 $10^{2 * 10^6}$ 筆操作,現行的超級電腦約可以達到每秒 $10^12$ 左右的運算次數,一世紀大約有 $3 * 10^9$ 秒,我們大約需要經過 $10^{2 * 10^6}$ 世紀後才有機會跑完這份 code。這時候,我們可以考慮如下的程式碼:
\begin{C++}
	int f(int n) {
		int arr[10000005];
		arr[1] = 1;
		for (int i = 2; i <= n; i++) {
			arr[i] = arr[i-1] + arr[i-2];
		}
		return arr[n];
	}
\end{C++}
上述程式碼的複雜度為 $O(n)$ ,倘若要計算 $f(10^7)$,以目前大家手邊的電腦,也僅需要不到一秒即可輸出結果。\\
上述的例子是 DP 最基礎的應用,欲先算好每個子問題的答案並紀錄之,便可以迅速求得母問題的答案,達到「以空間換取時間」的效果,這個概念十分抽象,但應用卻非常廣泛,講義後面會舉出更多例題供大家參考。

\subsection{滾動DP} 
有時候我們會發現,有些子問題的答案在某個時間後就再也不會被用到了,這時候我們就可以將新的答案覆蓋上去,進而達到節省空間的效果。一樣用費氏數列來舉例,請大家看看如下程式碼:
\begin{C++}
	int f(int n) {
		int a, b, c;
		a = b = 1;
		for (int i = 3; i <= n; i++) {
			c = a + b;
			a = b;
			b = c;
		}
		return c;
	}
\end{C++}
這樣便完成了時間 $O(n)$、空間 $O(1)$ 的費氏數列演算法。而事實上,已知費氏數列的最佳演算法複雜度是 $O(\log n)$ ,詳情請參考矩陣快速冪的部分。

\subsection{例題討論} 
接下來會講解一些最基本的 DP 問題~
\problem{最大連續和 (No Judge)}{
	給定一個長度為 $n$ 的序列 $a_1, a_2, ... , a_n$ ,求出這個序列中最大的區間和。
}
建立一個陣列 arr,arr[$i$] 代表以 $a_i$ 為右界所能產生出的最大連續和,可以列出轉移式 arr[$i$] $=$ max(arr[$i-1$]$+ a_i$, $a_i$),而由於第 $i$ 項的答案最多只需用到第 $i-1$ 項的答案,因此可以利用滾動法來實作。
\begin{C++}
	int f(int n, int a[]) {
		int now; // 表示以當前指標為右界的答案
		int ans; // 表示最終答案
		for (int i = 1; i <= n; i++) {
			now = max(now + a[i], a[i]);
			ans = max(ans, now);
		}
		return ans;
	}
\end{C++}

\problem{背包問題 (No Judge)}{
	你有一個背包,裡頭最多能裝重量總和為 $W$ 的物品,而你有 $N$ 個物品,第$i$項物品分別有價值 $v_i$ 與重量 $w_i$,請求出在不超過背包重量限制的情況下,最多能夠拿到多少價值的物品。
}
建立一個陣列arr,arr[$i$] 代表當重量為 $i$ 時所能裝下的最大價值,接著分別利用每個物品來進行轉移,  arr[$i$] $=$ max(arr[$i-w_i$]$+ v_i$, arr[$i$]),而過程中為大的 arr[$i$] 就是我們所要的答案。
\begin{C++}
	int f(int n, int W, item C[]) {
		int arr[MAXN+5], ans;
		for (int i = 0; i < n; i++) {
			for (int j = W; j >= C[i].w; j--) {
				arr[j] = max(arr[j], arr[j-C[i].w] + C[i].v);
				ans = max(ans, arr[j]);
			}
		}
		return ans;
	}
\end{C++}
\subsection{習題} 
來寫習題囉
\problem{無限背包問題 (No Judge)}{
	你有一個背包,裡頭最多能裝重量總和為 $W$ 的物品,而你有 $N$ 種物品,每種物品分別有價值 $v$ 與重量 $w$,而數量則沒有限制,請求出在不超過背包重量限制的情況下,最多能夠拿到多少價值的物品。
}
\problem{進化版最大連續和 (CF 1155D)}{
	一樣是求最大連續和,只是你多了一個操作,可以將某個區間 [$l, r$] 的值都同乘以 $k$,且只能做一次。( $n \leq 3 \cdot 10^5$ )
}

\problem{Increasing by Modulo(CF 1169C)}{
	給定一個長度為$n$的序列$a_1, a_2, a_3, \dots a_n$,和一個數字$m$,已知每一個數字滿足$0 \leq a_i < m$。每一次運算,蕭蕭會選擇若干個元素,編號分別為$i_1, i_2, \dots, i_k$,而將每一個$a_{i_j}$換成$(a_{i_j} + 1) \pmod{m}$。請問蕭蕭最少需要做幾次這個運算,才能使這個序列為非嚴格遞增?($1 \leq n, m \leq 300000$)
}