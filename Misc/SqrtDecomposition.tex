\chapter{根號算法}
\section{前言}
	「十則圍之、五則攻之、倍則分之」,這是孫子兵法中提到的,說要視面對敵人的多寡,要採用不同的策略。同樣地,如果題目給的範圍有不同的限制,可能需要不同的作法。有時,這個限制可可能會在題目中就給了;不過,有時候題目連這個都不會給!所以,就必須要自己來界定分離的標準了。這個方法有時候看起來很通靈,甚至很唬爛,但是經過分析之後是對的,時常能將複雜度的一個變數降為其根號,譬如一個$O(N^2)$的演算法其實經過一點巧思是$O(N\sqrt{N})$,雖然會比$\log$等級的演算法慢一些,但是通常實作複雜度低許多,是一個很實用的工具!

\section{種類壓大小}
	還是一樣,先看一個簡單的例題吧:
	\problem{詭異的詢問(No Judge)}{
		我有一個神秘函數$f(x)$。每次,你都可以詢問一個$x$,我就會回你$f(x)$的值。另外一個人會問你$Q \leq 10^6$個正整數$k_i$,代表你要回答說$f(k)$等於多少,而且保證$\sum^{Q - 1}_{i = 0} k_i \leq 10^8$。然而,我很不耐煩,所以你只能問我$1000$次的詢問。請好好利用這$15000$次的詢問,來回答另外一個人的$10^6$個詢問吧!
	}

	乍看之下,這題根本不可做——有$10^6$筆詢問,可能有$10^8$個不同的數字,而我只有$10^4$次的詢問,根本不可做!首先,會先想到說如果重複詢問的話,那就先把問過的東西存下來再回答就好啦!所以呢,現在的問題就變成:到底有多少個可能的詢問呢?不過,或許眼尖的你注意到了奇怪的限制:$\sum^{Q - 1}_{i = 0} k_i \leq 10^8$!這有什麼用呢?想要讓數字最多種,當然要讓詢問的數字越小越好,才比較不會超越所給的限制,才會有最多種的數字呀!所以,假設是有$n$種數字,則一定是依序詢問$1, 2, 3, \dots, n$才好,所以呢:
	\begin{align*}
		\frac{n(n + 1)}{2} &\leq 10^8\\	
		(n + 1)^2 &\leq 2 \times 10^8\\
		n &\leq \sqrt{2 \times 10^8} - 1 \approx 14141 < 15000
	\end{align*}
	也就是呢,最多有$14141$種相異個數字會出現,所以只要存下來就好了!運用這個技巧,可以推論出一個結論:
	\theorem{和與種類的不等式}{
		假設有一堆數字的和為$S$,則那些數字的種類數量的數量級為$O(\sqrt{S})$。
	}
	此定理的證明和上面是一樣的,這裏就不贅述了。

\section{不會有那麼多個吧!}
	在一個圖中尋找三角形是很經典的一個問題:
	\problem{尋找三角形(經典問題)}{
		給定一個有$N \leq 10^4$個點,$M \leq 10^4$個邊的圖,請問這個圖有幾個三角形?一個三角形的定義為一個無序的3-tuple $(u, v, w)$($1 \leq u, v, w \leq N$),使得那三個點兩兩有邊連接。
	}
	先直接丟唬爛的解:枚舉一個邊$(u, v)$,看$\text{deg}(u)$和$\text{deg}(v)$哪一個比較小(假設是$u$),那就直接硬枚舉所有與$u$連結的點$x$,並看$x$和$v$是否有沒有連結,如果有的話,就直接加一。這樣,每個三角形都會被數到三次,所以就輸出答案除以三就好了!這樣的複雜度是$O(NM)$,因為對於每一個邊,都有可能掃到每另外一個邊。所以呢,先寫看看,傳上去,可以拿個部分——什麼?!居然AC了?為什麼呀?一樣用根號算法中的分case方法來分析其複雜度。接下來我們稱度數小於$\sqrt{M}$的點作「輕點」,反之則稱為「重點」。接下來考慮每個點當「小點」(一條邊連接的兩個點中度數較小的那個)對複雜度的貢獻。首先如果是輕點當小點,那就太棒了!因為輕點的度數$\sqrt{M}$
