\chapter{字串\uppercase\expandafter{\romannumeral 2}}
\section{AC自動機(AC Automaton)}
AC自動機顧名思義就是一種可以讓你自動AC的程式啊(X\\

好,絕對不要聽別人這樣唬爛,事實上AC自動機是KMP字串匹配演算法的進階版,可以一次尋找很多目標字串各個在主字串中出現的地方。不知道大家還記不記得,KMP演算法是在進行字串匹配前,計算當匹配失敗時,可以直接從哪個位置開始重新匹配,也就是failure function。而在AC自動機中也是利用相同的手法,在目標字串們建構出的trie上計算failure pointer,同樣代表匹配失敗時應從哪裡重新開始。
\subsection{節點}
AC自動機的節點大致上與trie都一樣,只是會多新增幾個資料,詳細的\inline{struct}如下:
\begin{C++}
struct node{ //maxn means how many kinds of characters
	node *p,*next[maxn],*fp; //記錄父節點、子節點們和failure pointer
    int index; //第幾個單字的結尾,-1代表不是結尾
    char ch;
    node(){}
    node(node *_p,char _c):p(_p),ch(_c){
        for(int i=0;i<maxn;i++) next[i]=nullptr;
        fp=nullptr; index=-1;
    }
};
\end{C++}
\subsection{build AC automaton}
最重要的當然是先建構出trie才能做其他事吧!當然,建構方式和正常的trie也差不多,因此就直接看code吧:
\begin{C++}
node *build(vector<string> &dict){
    node *root=new node(nullptr,'\0');
    root->p=root;
    for(int i=0;i<dict.size();i++){
        node *ptr=root;
        for(auto &j:dict[i]){
            if(!ptr->next[j-'a'])
                ptr->next[j-'a']=new node(ptr,j);
            ptr=ptr->next[j-'a'];
        }
        ptr->index=i;
    }
    //failure pointer part
    return root;
}
\end{C++}
\subsection{failure pointer}
為了方便,這裡的failure pointer定義和failure function會有點差異。而建立的方法我們知道原本failure function在計算時可以\inline{for}迴圈從頭開始一一往後建,也就是說計算某個點時其前面所有failure function都要先被算到。因此移到trie時我們選用BFS的遍歷順序依序計算,如此便可保證計算某一節點時比他上面的節點都已經算完了。說了這麼多,不如直接看程式碼吧(這邊會接續上面的程式碼):
\begin{C++}
node *build(vector<string> &dict){
    //failure pointer part
    queue<node *> q;
    for(int i=0;i<maxn;i++){ //根及第一層的fp都指向根
        if(root->next[i]){
            root->next[i]->fp=root;
            for(int j=0;j<maxn;j++)
                if(root->next[i]->next[j])
                    q.push(root->next[i]->next[j]);
        }
    }
    while(q.size()){
        node *t=q.front(),*ptr=t->p->fp; q.pop();
        while(ptr!=root&&!ptr->next[t->ch-'a'])
            ptr=ptr->fp;
        if(ptr->next[t->ch-'a']) t->fp=ptr->next[t->ch-'a'];
        else t->fp=root;
        for(int i=0;i<maxn;i++) if(t->next[i]) q.push(t->next[i]);
    }
    return root;
}
\end{C++}
\subsection{開始匹配}
不外乎就是跟KMP的方法一樣,從第一個字開始看,若遇到失配的狀況就走failure pointer。程式碼應該也相當易懂:
\begin{C++}
void ACsearch(node *r,string &s,vector<string> &dict){
    //root of trie,input string and dictionary
    node *ptr=r;
    int i=0;
    cout<<"string:\tposition:"<<endl;
    while(i<s.size()){
        if(ptr->next[s[i]-'a'])
            ptr=ptr->next[s[i]-'a'],i++;
        else if(ptr==r)
            i++;
        else
            ptr=ptr->fp;
        if(ptr->index!=-1)
            cout<<dict[ptr->index]<<"\t"<<i-(int)dict[ptr->index].size()<<endl;
    }
    while(ptr!=r){
        ptr=ptr->fp;
        if(ptr->index!=-1)
        	cout<<dict[ptr->index]<<"\t"<<i-(int)dict[ptr->index].size()<<endl;
    } //makes ptr back to root
}
\end{C++}
\subsection{你以為結束了?}
多玩幾次之後就會發現上面AC自動機在目標字串互相包含時會出問題!看下面範例:\\

\inline{dict={"ab","cababc"}; s="cababc"}。上面程式的輸出結果ab只在位置$0$和$4$,位置$2$的ab消失了!!!為什麼會這樣?你畫出AC自動機後自己走一次就會發現,整個匹配過程都是走"ababab"的那條路,經過第一個ab時在tire上有標記結尾,經過第三個ab時,因為結尾加上的\inline{'$\setminus$n'}不斷失配而往fp走,走到指向第一個ab的地方時就會被算到。然而經過第二個ab時,完全沒有標記讓AC自動機知道他也是一個單字。也就是說,只要某個字被包含在另一個字的中間時,就會被忽略(若是頭尾還是會被算到)。怎麼辦呢???\\

假設從節點$S_{0,1\dots k-1}$走到$S_{0,1\dots k}$時,會使$S_{1,2\dots k-2}$的所有後綴字串沒有被考慮到。因此我們應該要瓊舉這些字串,檢查他們是不是也在字典中。然而,我們可以發現$S_{1,2\dots k-2}$後綴字串中所有在字典裡的字串,事實上就是不斷走failure pointer路上經過的那些字串!想一想就知道了!\\

因此,實作上我們可以記錄所有點被走過幾次,然後在ACsearch跑完之後將每個節點被算到的次數不斷向他的failure pointer加。那應該用什麼順序處理這件事呢?因為要不斷向上推,不如就從最深的開始處理(用DFS就好了),如此就可以正確又高效率的完成了!所以來看一下code吧!(由於互相包含版紀錄位置很麻煩,我也不知道複雜度是不是好的,所以這邊就放計數就好了)
\begin{C++}
待補
\end{C++}