\chapter{樹DP}
    \section{前言}
    顧名思義,就是在樹上做 DP。
    \section{例子}
        \subsection{算子樹大小}
        一個點的子樹大小就是在他底下(包含自己)的節點數。\\
        定義 $dp_u$ 為 $u$ 的子樹大小。\\
        $dp_u = 1 + \sum\limits_{v\ is \ u's\ son}{dp_v}$ \\
        可以從根往下 DFS,每次回傳自己的 dp值給爸爸加。\\
        \begin{C++}
        int DP(int u /*現在的點*/, int f /*爸爸*/){
            dp[u] = 1; //自己佔一個
            for(int v : T[u]){ //枚舉所有鄰居
                if(v == f) continue; //好馬不吃回頭草
                dp[u] += DP(v, u); //往下遞迴
            }
            return dp[u]; //回傳自己的子樹大小
        }
        \end{C++}
        因為每個節點只會被走過一次,所以就可以在 $O(N)$ 的時間做出來了。
        \subsection{樹直徑}
        樹直徑 = 樹上相隔最遠的兩節點之間距離。
